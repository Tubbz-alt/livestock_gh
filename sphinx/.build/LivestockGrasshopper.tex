%% Generated by Sphinx.
\def\sphinxdocclass{report}
\documentclass[letterpaper,10pt,english]{sphinxmanual}
\ifdefined\pdfpxdimen
   \let\sphinxpxdimen\pdfpxdimen\else\newdimen\sphinxpxdimen
\fi \sphinxpxdimen=.75bp\relax

\usepackage[utf8]{inputenc}
\ifdefined\DeclareUnicodeCharacter
 \ifdefined\DeclareUnicodeCharacterAsOptional
  \DeclareUnicodeCharacter{"00A0}{\nobreakspace}
  \DeclareUnicodeCharacter{"2500}{\sphinxunichar{2500}}
  \DeclareUnicodeCharacter{"2502}{\sphinxunichar{2502}}
  \DeclareUnicodeCharacter{"2514}{\sphinxunichar{2514}}
  \DeclareUnicodeCharacter{"251C}{\sphinxunichar{251C}}
  \DeclareUnicodeCharacter{"2572}{\textbackslash}
 \else
  \DeclareUnicodeCharacter{00A0}{\nobreakspace}
  \DeclareUnicodeCharacter{2500}{\sphinxunichar{2500}}
  \DeclareUnicodeCharacter{2502}{\sphinxunichar{2502}}
  \DeclareUnicodeCharacter{2514}{\sphinxunichar{2514}}
  \DeclareUnicodeCharacter{251C}{\sphinxunichar{251C}}
  \DeclareUnicodeCharacter{2572}{\textbackslash}
 \fi
\fi
\usepackage{cmap}
\usepackage[T1]{fontenc}
\usepackage{amsmath,amssymb,amstext}
\usepackage{babel}
\usepackage{times}
\usepackage[Bjarne]{fncychap}
\usepackage[dontkeepoldnames]{sphinx}

\usepackage{geometry}

% Include hyperref last.
\usepackage{hyperref}
% Fix anchor placement for figures with captions.
\usepackage{hypcap}% it must be loaded after hyperref.
% Set up styles of URL: it should be placed after hyperref.
\urlstyle{same}

\addto\captionsenglish{\renewcommand{\figurename}{Fig.}}
\addto\captionsenglish{\renewcommand{\tablename}{Table}}
\addto\captionsenglish{\renewcommand{\literalblockname}{Listing}}

\addto\captionsenglish{\renewcommand{\literalblockcontinuedname}{continued from previous page}}
\addto\captionsenglish{\renewcommand{\literalblockcontinuesname}{continues on next page}}

\addto\extrasenglish{\def\pageautorefname{page}}

\setcounter{tocdepth}{1}



\title{Livestock Grasshopper Documentation}
\date{Mar 07, 2018}
\release{2018.03}
\author{Christian Kongsgaard}
\newcommand{\sphinxlogo}{\vbox{}}
\renewcommand{\releasename}{Release}
\makeindex

\begin{document}

\maketitle
\sphinxtableofcontents
\phantomsection\label{\detokenize{index::doc}}


Livestock is the name of the library of components that has been developed for this thesis.
Livestock consists of a series of Grasshopper Python Script components and a underlying collection of Python scripts
and a PyPI \textendash{} Python Package Index - package. This is the documentation for the Grasshopper package.


\chapter{Documentation for the Grasshopper Package:}
\label{\detokenize{index:documentation-for-the-grasshopper-package}}\label{\detokenize{index:welcome-to-livestock-gh-s-documentation}}

\section{Livestock Grasshopper Components}
\label{\detokenize{components:components}}\label{\detokenize{components:livestock-grasshopper-components}}\label{\detokenize{components::doc}}

\subsection{0 \textbar{} Miscellaneous}
\label{\detokenize{components:miscellaneous}}
\sphinxstylestrong{Livestock Python Executor}
\begin{quote}\begin{description}
\item[{Description}] \leavevmode
\begin{DUlineblock}{0em}
\item[] Path to python executor.
\end{DUlineblock}

\item[{Inputs}] \leavevmode\begin{quote}\begin{description}
\item[{1.}] \leavevmode\begin{quote}\begin{description}
\item[{Name}] \leavevmode
PythonPath

\item[{Description}] \leavevmode
Path to python.exe

\item[{Data Access}] \leavevmode
Item

\item[{Default Value}] \leavevmode
\begin{DUlineblock}{0em}
\item[] None
\end{DUlineblock}

\end{description}\end{quote}

\end{description}\end{quote}

\item[{Outputs}] \leavevmode\begin{quote}\begin{description}
\item[{1.}] \leavevmode\begin{quote}\begin{description}
\item[{Name}] \leavevmode
readMe!

\item[{Description}] \leavevmode
\begin{DUlineblock}{0em}
\item[] In case of any errors, it will be shown here.
\end{DUlineblock}

\end{description}\end{quote}

\item[{2.}] \leavevmode\begin{quote}\begin{description}
\item[{Name}] \leavevmode
BoundaryCondition

\item[{Description}] \leavevmode
\begin{DUlineblock}{0em}
\item[] Livestock Boundary Conditions.
\end{DUlineblock}

\end{description}\end{quote}

\end{description}\end{quote}

\end{description}\end{quote}

\sphinxstylestrong{Livestock SSH Connection}
\begin{quote}\begin{description}
\item[{Description}] \leavevmode
\begin{DUlineblock}{0em}
\item[] Setup SSH connection.
\item[] Icon based on art from Arthur Shlain from the Noun Project.
\end{DUlineblock}

\item[{Inputs}] \leavevmode\begin{quote}\begin{description}
\item[{1.}] \leavevmode\begin{quote}\begin{description}
\item[{Name}] \leavevmode
IP

\item[{Description}] \leavevmode
IP Address for SSH connection.

\item[{Data Access}] \leavevmode
Item

\item[{Default Value}] \leavevmode
\begin{DUlineblock}{0em}
\item[] None
\end{DUlineblock}

\end{description}\end{quote}

\item[{2.}] \leavevmode\begin{quote}\begin{description}
\item[{Name}] \leavevmode
Port

\item[{Description}] \leavevmode
Port for SSH connection.

\item[{Data Access}] \leavevmode
Item

\item[{Default Value}] \leavevmode
\begin{DUlineblock}{0em}
\item[] None
\end{DUlineblock}

\end{description}\end{quote}

\item[{3.}] \leavevmode\begin{quote}\begin{description}
\item[{Name}] \leavevmode
Username

\item[{Description}] \leavevmode
Username for SSH connection.

\item[{Data Access}] \leavevmode
Item

\item[{Default Value}] \leavevmode
\begin{DUlineblock}{0em}
\item[] None
\end{DUlineblock}

\end{description}\end{quote}

\item[{4.}] \leavevmode\begin{quote}\begin{description}
\item[{Name}] \leavevmode
Password

\item[{Description}] \leavevmode
Password for SSH connection.

\item[{Data Access}] \leavevmode
Item

\item[{Default Value}] \leavevmode
\begin{DUlineblock}{0em}
\item[] None
\end{DUlineblock}

\end{description}\end{quote}

\end{description}\end{quote}

\item[{Outputs}] \leavevmode\begin{quote}\begin{description}
\item[{1.}] \leavevmode\begin{quote}\begin{description}
\item[{Name}] \leavevmode
readMe!

\item[{Description}] \leavevmode
\begin{DUlineblock}{0em}
\item[] In case of any errors, it will be shown here.
\end{DUlineblock}

\end{description}\end{quote}

\end{description}\end{quote}

\end{description}\end{quote}


\subsection{1 \textbar{} Geometry}
\label{\detokenize{components:geometry}}
\sphinxstylestrong{Livestock Load Mesh}
\begin{quote}\begin{description}
\item[{Description}] \leavevmode
Loads a mesh.

\item[{Inputs}] \leavevmode\begin{quote}\begin{description}
\item[{1.}] \leavevmode\begin{quote}\begin{description}
\item[{Name}] \leavevmode
Filename

\item[{Description}] \leavevmode
Directory and file name of mesh.

\item[{Data Access}] \leavevmode
Item

\item[{Default Value}] \leavevmode
\begin{DUlineblock}{0em}
\item[] None
\end{DUlineblock}

\end{description}\end{quote}

\item[{2.}] \leavevmode\begin{quote}\begin{description}
\item[{Name}] \leavevmode
Load

\item[{Description}] \leavevmode
Activates the component.

\item[{Data Access}] \leavevmode
Item

\item[{Default Value}] \leavevmode
\begin{DUlineblock}{0em}
\item[] False
\end{DUlineblock}

\end{description}\end{quote}

\end{description}\end{quote}

\item[{Outputs}] \leavevmode\begin{quote}\begin{description}
\item[{1.}] \leavevmode\begin{quote}\begin{description}
\item[{Name}] \leavevmode
readMe!

\item[{Description}] \leavevmode
\begin{DUlineblock}{0em}
\item[] In case of any errors, it will be shown here.
\end{DUlineblock}

\end{description}\end{quote}

\item[{2.}] \leavevmode\begin{quote}\begin{description}
\item[{Name}] \leavevmode
Mesh

\item[{Description}] \leavevmode
\begin{DUlineblock}{0em}
\item[] Loaded mesh.
\end{DUlineblock}

\end{description}\end{quote}

\item[{3.}] \leavevmode\begin{quote}\begin{description}
\item[{Name}] \leavevmode
MeshData

\item[{Description}] \leavevmode
\begin{DUlineblock}{0em}
\item[] Additional data if any.
\end{DUlineblock}

\end{description}\end{quote}

\end{description}\end{quote}

\end{description}\end{quote}

\sphinxstylestrong{Livestock Save Mesh}
\begin{quote}\begin{description}
\item[{Description}] \leavevmode
Saves a mesh and additional data

\item[{Inputs}] \leavevmode\begin{quote}\begin{description}
\item[{1.}] \leavevmode\begin{quote}\begin{description}
\item[{Name}] \leavevmode
Mesh

\item[{Description}] \leavevmode
Mesh to save.

\item[{Data Access}] \leavevmode
Item

\item[{Default Value}] \leavevmode
\begin{DUlineblock}{0em}
\item[] None
\end{DUlineblock}

\end{description}\end{quote}

\item[{2.}] \leavevmode\begin{quote}\begin{description}
\item[{Name}] \leavevmode
Data

\item[{Description}] \leavevmode
Additional data if any.

\item[{Data Access}] \leavevmode
Item

\item[{Default Value}] \leavevmode
\begin{DUlineblock}{0em}
\item[] None
\end{DUlineblock}

\end{description}\end{quote}

\item[{3.}] \leavevmode\begin{quote}\begin{description}
\item[{Name}] \leavevmode
Directory

\item[{Description}] \leavevmode
File path to save mesh to.

\item[{Data Access}] \leavevmode
Item

\item[{Default Value}] \leavevmode
\begin{DUlineblock}{0em}
\item[] None
\end{DUlineblock}

\end{description}\end{quote}

\item[{4.}] \leavevmode\begin{quote}\begin{description}
\item[{Name}] \leavevmode
Filename

\item[{Description}] \leavevmode
File name.

\item[{Data Access}] \leavevmode
Item

\item[{Default Value}] \leavevmode
\begin{DUlineblock}{0em}
\item[] None
\end{DUlineblock}

\end{description}\end{quote}

\item[{5.}] \leavevmode\begin{quote}\begin{description}
\item[{Name}] \leavevmode
Save

\item[{Description}] \leavevmode
Activates the component.

\item[{Data Access}] \leavevmode
Item

\item[{Default Value}] \leavevmode
\begin{DUlineblock}{0em}
\item[] False
\end{DUlineblock}

\end{description}\end{quote}

\end{description}\end{quote}

\item[{Outputs}] \leavevmode\begin{quote}\begin{description}
\item[{1.}] \leavevmode\begin{quote}\begin{description}
\item[{Name}] \leavevmode
readMe!

\item[{Description}] \leavevmode
\begin{DUlineblock}{0em}
\item[] In case of any errors, it will be shown here.
\end{DUlineblock}

\end{description}\end{quote}

\end{description}\end{quote}

\end{description}\end{quote}


\subsection{3 \textbar{} CMF}
\label{\detokenize{components:cmf}}
\sphinxstylestrong{Livestock CMF Ground}
\begin{quote}\begin{description}
\item[{Description}] \leavevmode
\begin{DUlineblock}{0em}
\item[] Generates CMF ground.
\item[] Icon art based created by Ben Davis from the Noun Project.
\end{DUlineblock}

\item[{Inputs}] \leavevmode\begin{quote}\begin{description}
\item[{1.}] \leavevmode\begin{quote}\begin{description}
\item[{Name}] \leavevmode
Layers

\item[{Description}] \leavevmode
Soil layers to add to the mesh in m.

\item[{Data Access}] \leavevmode
Item

\item[{Default Value}] \leavevmode
\begin{DUlineblock}{0em}
\item[] 0
\end{DUlineblock}

\end{description}\end{quote}

\item[{2.}] \leavevmode\begin{quote}\begin{description}
\item[{Name}] \leavevmode
RetentionCurve

\item[{Description}] \leavevmode
Livestock CMF Retention Curve.

\item[{Data Access}] \leavevmode
Item

\item[{Default Value}] \leavevmode
\begin{DUlineblock}{0em}
\item[] None
\end{DUlineblock}

\end{description}\end{quote}

\item[{3.}] \leavevmode\begin{quote}\begin{description}
\item[{Name}] \leavevmode
VegetationProperties

\item[{Description}] \leavevmode
Input from Livestock CMF Vegetation Properties.

\item[{Data Access}] \leavevmode
Item

\item[{Default Value}] \leavevmode
\begin{DUlineblock}{0em}
\item[] None
\end{DUlineblock}

\end{description}\end{quote}

\item[{4.}] \leavevmode\begin{quote}\begin{description}
\item[{Name}] \leavevmode
SaturatedDepth

\item[{Description}] \leavevmode
Initial saturated depth in m. It is depth where the groundwater is located. Default is set
to 3m.

\item[{Data Access}] \leavevmode
Item

\item[{Default Value}] \leavevmode
\begin{DUlineblock}{0em}
\item[] 3
\end{DUlineblock}

\end{description}\end{quote}

\item[{5.}] \leavevmode\begin{quote}\begin{description}
\item[{Name}] \leavevmode
FaceIndices

\item[{Description}] \leavevmode
List of face indices, on where the ground properties are applied.

\item[{Data Access}] \leavevmode
List

\item[{Default Value}] \leavevmode
\begin{DUlineblock}{0em}
\item[] None
\end{DUlineblock}

\end{description}\end{quote}

\item[{6.}] \leavevmode\begin{quote}\begin{description}
\item[{Name}] \leavevmode
ETMethod

\item[{Description}] \leavevmode
\begin{DUlineblock}{0em}
\item[] Set method to calculate evapotranspiration.
\item[] 0: No evapotranspiration.
\item[] 1: Penman-Monteith.
\item[] 2: Shuttleworth-Wallace.
\item[] Default is set to Shuttleworth-Wallace.
\end{DUlineblock}

\item[{Data Access}] \leavevmode
Item

\item[{Default Value}] \leavevmode
\begin{DUlineblock}{0em}
\item[] 2
\end{DUlineblock}

\end{description}\end{quote}

\item[{7.}] \leavevmode\begin{quote}\begin{description}
\item[{Name}] \leavevmode
Manning

\item[{Description}] \leavevmode
Set Manning roughness. If not set CMF calculates it from the above given values.

\item[{Data Access}] \leavevmode
Item

\item[{Default Value}] \leavevmode
\begin{DUlineblock}{0em}
\item[] None
\end{DUlineblock}

\end{description}\end{quote}

\item[{8.}] \leavevmode\begin{quote}\begin{description}
\item[{Name}] \leavevmode
PuddleDepth

\item[{Description}] \leavevmode
Set puddle depth. Puddle depth is the height were run-off begins.

\item[{Data Access}] \leavevmode
Item

\item[{Default Value}] \leavevmode
\begin{DUlineblock}{0em}
\item[] 0.01
\end{DUlineblock}

\end{description}\end{quote}

\end{description}\end{quote}

\item[{Outputs}] \leavevmode\begin{quote}\begin{description}
\item[{1.}] \leavevmode\begin{quote}\begin{description}
\item[{Name}] \leavevmode
readMe!

\item[{Description}] \leavevmode
In case of any errors, it will be shown here.

\end{description}\end{quote}

\item[{2.}] \leavevmode\begin{quote}\begin{description}
\item[{Name}] \leavevmode
Ground

\item[{Description}] \leavevmode
Livestock Ground Data Class.

\end{description}\end{quote}

\end{description}\end{quote}

\end{description}\end{quote}

\sphinxstylestrong{Livestock CMF Weather}
\begin{quote}\begin{description}
\item[{Description}] \leavevmode
\begin{DUlineblock}{0em}
\item[] Generates CMF weather.
\item[] Icon art based created by Adrien Coquet from the Noun Project.
\end{DUlineblock}

\item[{Inputs}] \leavevmode\begin{quote}\begin{description}
\item[{1.}] \leavevmode\begin{quote}\begin{description}
\item[{Name}] \leavevmode
Temperature

\item[{Description}] \leavevmode
Temperature in C. Either a list or a tree where the number of branches is equal to the number
of mesh faces.

\item[{Data Access}] \leavevmode
Tree

\item[{Default Value}] \leavevmode
\begin{DUlineblock}{0em}
\item[] None
\end{DUlineblock}

\end{description}\end{quote}

\item[{2.}] \leavevmode\begin{quote}\begin{description}
\item[{Name}] \leavevmode
WindSpeed

\item[{Description}] \leavevmode
Wind speed in m/s. Either a list or a tree where the number of branches is equal to the number
of mesh faces.

\item[{Data Access}] \leavevmode
Tree

\item[{Default Value}] \leavevmode
\begin{DUlineblock}{0em}
\item[] None
\end{DUlineblock}

\end{description}\end{quote}

\item[{3.}] \leavevmode\begin{quote}\begin{description}
\item[{Name}] \leavevmode
RelativeHumidity

\item[{Description}] \leavevmode
Relative humidity in \%. Either a list or a tree where the number of branches is equal to the number
of mesh faces.

\item[{Data Access}] \leavevmode
Tree

\item[{Default Value}] \leavevmode
\begin{DUlineblock}{0em}
\item[] None
\end{DUlineblock}

\end{description}\end{quote}

\item[{4.}] \leavevmode\begin{quote}\begin{description}
\item[{Name}] \leavevmode
CloudCover

\item[{Description}] \leavevmode
Cloud cover, unitless between 0 and 1. Either a list or a tree where the number of branches is equal to the number
of mesh faces.

\item[{Data Access}] \leavevmode
Tree

\item[{Default Value}] \leavevmode
\begin{DUlineblock}{0em}
\item[] None
\end{DUlineblock}

\end{description}\end{quote}

\item[{5.}] \leavevmode\begin{quote}\begin{description}
\item[{Name}] \leavevmode
GlobalRadiation

\item[{Description}] \leavevmode
Global Radiation in W/m:sup:\sphinxtitleref{2}. Either a list or a tree where the number of branches is equal to the number
of mesh faces.

\item[{Data Access}] \leavevmode
Tree

\item[{Default Value}] \leavevmode
\begin{DUlineblock}{0em}
\item[] None
\end{DUlineblock}

\end{description}\end{quote}

\item[{6.}] \leavevmode\begin{quote}\begin{description}
\item[{Name}] \leavevmode
Rain

\item[{Description}] \leavevmode
Horizontal precipitation in mm/h. Either a list or a tree where the number of branches is equal to the number
of mesh faces.

\item[{Data Access}] \leavevmode
Tree

\item[{Default Value}] \leavevmode
\begin{DUlineblock}{0em}
\item[] None
\end{DUlineblock}

\end{description}\end{quote}

\item[{7.}] \leavevmode\begin{quote}\begin{description}
\item[{Name}] \leavevmode
GroundTemperature

\item[{Description}] \leavevmode
Ground temperature in C. Either a list or a tree where the number of branches is equal to the number
of mesh faces.

\item[{Data Access}] \leavevmode
Tree

\item[{Default Value}] \leavevmode
\begin{DUlineblock}{0em}
\item[] None
\end{DUlineblock}

\end{description}\end{quote}

\item[{8.}] \leavevmode\begin{quote}\begin{description}
\item[{Name}] \leavevmode
Location

\item[{Description}] \leavevmode
A Ladybug Tools Locations.

\item[{Data Access}] \leavevmode
Item

\item[{Default Value}] \leavevmode
\begin{DUlineblock}{0em}
\item[] None
\end{DUlineblock}

\end{description}\end{quote}

\item[{9.}] \leavevmode\begin{quote}\begin{description}
\item[{Name}] \leavevmode
MeshFaceCount

\item[{Description}] \leavevmode
Number of faces in the ground mesh.

\item[{Data Access}] \leavevmode
Item

\item[{Default Value}] \leavevmode
\begin{DUlineblock}{0em}
\item[] None
\end{DUlineblock}

\end{description}\end{quote}

\end{description}\end{quote}

\item[{Outputs}] \leavevmode\begin{quote}\begin{description}
\item[{1.}] \leavevmode\begin{quote}\begin{description}
\item[{Name}] \leavevmode
readMe!

\item[{Description}] \leavevmode
\begin{DUlineblock}{0em}
\item[] In case of any errors, it will be shown here.
\end{DUlineblock}

\end{description}\end{quote}

\item[{2.}] \leavevmode\begin{quote}\begin{description}
\item[{Name}] \leavevmode
Weather

\item[{Description}] \leavevmode
\begin{DUlineblock}{0em}
\item[] Livestock Weather Data Class.
\end{DUlineblock}

\end{description}\end{quote}

\end{description}\end{quote}

\end{description}\end{quote}

\sphinxstylestrong{Livestock CMF Vegetation Properties}
\begin{quote}\begin{description}
\item[{Description}] \leavevmode
\begin{DUlineblock}{0em}
\item[] Generates CMF Vegetation Properties
\item[] Icon art based created by Ben Davis from the Noun Project.
\end{DUlineblock}

\item[{Inputs}] \leavevmode\begin{quote}\begin{description}
\item[{1.}] \leavevmode\begin{quote}\begin{description}
\item[{Name}] \leavevmode
Property

\item[{Description}] \leavevmode
0-1 grasses. 2-6 soils. Default is set to 0

\item[{Data Access}] \leavevmode
Item

\item[{Default Value}] \leavevmode
\begin{DUlineblock}{0em}
\item[] 0
\end{DUlineblock}

\end{description}\end{quote}

\end{description}\end{quote}

\item[{Outputs}] \leavevmode\begin{quote}\begin{description}
\item[{1.}] \leavevmode\begin{quote}\begin{description}
\item[{Name}] \leavevmode
readMe!

\item[{Description}] \leavevmode
\begin{DUlineblock}{0em}
\item[] In case of any errors, it will be shown here.
\end{DUlineblock}

\end{description}\end{quote}

\item[{2.}] \leavevmode\begin{quote}\begin{description}
\item[{Name}] \leavevmode
Units

\item[{Description}] \leavevmode
\begin{DUlineblock}{0em}
\item[] Shows the units of the surface values.
\end{DUlineblock}

\end{description}\end{quote}

\item[{3.}] \leavevmode\begin{quote}\begin{description}
\item[{Name}] \leavevmode
VegetationValues

\item[{Description}] \leavevmode
\begin{DUlineblock}{0em}
\item[] Chosen vegetation property values.
\end{DUlineblock}

\end{description}\end{quote}

\item[{4.}] \leavevmode\begin{quote}\begin{description}
\item[{Name}] \leavevmode
VegetationProperties

\item[{Description}] \leavevmode
\begin{DUlineblock}{0em}
\item[] Livestock Vegetation Property Data.
\end{DUlineblock}

\end{description}\end{quote}

\end{description}\end{quote}

\end{description}\end{quote}

\sphinxstylestrong{Livestock CMF Synthetic Tree}
\begin{quote}\begin{description}
\item[{Description}] \leavevmode
\begin{DUlineblock}{0em}
\item[] Generates a synthetic tree
\end{DUlineblock}

\item[{Inputs}] \leavevmode\begin{quote}\begin{description}
\item[{1.}] \leavevmode\begin{quote}\begin{description}
\item[{Name}] \leavevmode
FaceIndex

\item[{Description}] \leavevmode
Mesh face index where tree is placed

\item[{Data Access}] \leavevmode
Item

\item[{Default Value}] \leavevmode
\begin{DUlineblock}{0em}
\item[] None
\end{DUlineblock}

\end{description}\end{quote}

\item[{2.}] \leavevmode\begin{quote}\begin{description}
\item[{Name}] \leavevmode
TreeType

\item[{Description}] \leavevmode
Tree types: 0 - Deciduous, 1 - Coniferous, 2 - Shrubs. Default is deciduous.

\item[{Data Access}] \leavevmode
Item

\item[{Default Value}] \leavevmode
\begin{DUlineblock}{0em}
\item[] 0
\end{DUlineblock}

\end{description}\end{quote}

\item[{3.}] \leavevmode\begin{quote}\begin{description}
\item[{Name}] \leavevmode
Height

\item[{Description}] \leavevmode
Height of tree in meters. Default is set to 10m

\item[{Data Access}] \leavevmode
Item

\item[{Default Value}] \leavevmode
\begin{DUlineblock}{0em}
\item[] 10
\end{DUlineblock}

\end{description}\end{quote}

\end{description}\end{quote}

\item[{Outputs}] \leavevmode\begin{quote}\begin{description}
\item[{1.}] \leavevmode\begin{quote}\begin{description}
\item[{Name}] \leavevmode
readMe!

\item[{Description}] \leavevmode
\begin{DUlineblock}{0em}
\item[] In case of any errors, it will be shown here.
\end{DUlineblock}

\end{description}\end{quote}

\item[{2.}] \leavevmode\begin{quote}\begin{description}
\item[{Name}] \leavevmode
Units

\item[{Description}] \leavevmode
\begin{DUlineblock}{0em}
\item[] Shows the units of the tree values.
\end{DUlineblock}

\end{description}\end{quote}

\item[{3.}] \leavevmode\begin{quote}\begin{description}
\item[{Name}] \leavevmode
TreeValues

\item[{Description}] \leavevmode
\begin{DUlineblock}{0em}
\item[] Chosen tree properties values.
\end{DUlineblock}

\end{description}\end{quote}

\item[{4.}] \leavevmode\begin{quote}\begin{description}
\item[{Name}] \leavevmode
TreeProperties

\item[{Description}] \leavevmode
\begin{DUlineblock}{0em}
\item[] Livestock tree properties data.
\end{DUlineblock}

\end{description}\end{quote}

\end{description}\end{quote}

\end{description}\end{quote}

\sphinxstylestrong{Livestock CMF Retention Curve}
\begin{quote}\begin{description}
\item[{Description}] \leavevmode
Generates a retention curve.

\item[{Inputs}] \leavevmode\begin{quote}\begin{description}
\item[{1.}] \leavevmode\begin{quote}\begin{description}
\item[{Name}] \leavevmode
SoilIndex

\item[{Description}] \leavevmode
Index for choosing soil type. Index from 0-5. Default is set to 0, which is the default CMF
retention curve.

\item[{Data Access}] \leavevmode
Item

\item[{Default Value}] \leavevmode
\begin{DUlineblock}{0em}
\item[] 0
\end{DUlineblock}

\end{description}\end{quote}

\item[{2.}] \leavevmode\begin{quote}\begin{description}
\item[{Name}] \leavevmode
K\_sat

\item[{Description}] \leavevmode
Saturated conductivity in m/day.

\item[{Data Access}] \leavevmode
Item

\item[{Default Value}] \leavevmode
\begin{DUlineblock}{0em}
\item[] None
\end{DUlineblock}

\end{description}\end{quote}

\item[{3.}] \leavevmode\begin{quote}\begin{description}
\item[{Name}] \leavevmode
Phi

\item[{Description}] \leavevmode
Porosity in m3/m3.

\item[{Data Access}] \leavevmode
Item

\item[{Default Value}] \leavevmode
\begin{DUlineblock}{0em}
\item[] None
\end{DUlineblock}

\end{description}\end{quote}

\item[{4.}] \leavevmode\begin{quote}\begin{description}
\item[{Name}] \leavevmode
Alpha

\item[{Description}] \leavevmode
Inverse of water entry potential in 1/cm.

\item[{Data Access}] \leavevmode
Item

\item[{Default Value}] \leavevmode
\begin{DUlineblock}{0em}
\item[] 0
\end{DUlineblock}

\end{description}\end{quote}

\item[{5.}] \leavevmode\begin{quote}\begin{description}
\item[{Name}] \leavevmode
N

\item[{Description}] \leavevmode
Pore size distribution parameter is unitless.

\item[{Data Access}] \leavevmode
Item

\item[{Default Value}] \leavevmode
\begin{DUlineblock}{0em}
\item[] None
\end{DUlineblock}

\end{description}\end{quote}

\item[{6.}] \leavevmode\begin{quote}\begin{description}
\item[{Name}] \leavevmode
M

\item[{Description}] \leavevmode
VanGenuchten m (if negative, 1-1/n is used) is unitless.

\item[{Data Access}] \leavevmode
Item

\item[{Default Value}] \leavevmode
\begin{DUlineblock}{0em}
\item[] None
\end{DUlineblock}

\end{description}\end{quote}

\item[{6.}] \leavevmode\begin{quote}\begin{description}
\item[{Name}] \leavevmode
L

\item[{Description}] \leavevmode
Mualem tortoisivity is unitless.

\item[{Data Access}] \leavevmode
Item

\item[{Default Value}] \leavevmode
\begin{DUlineblock}{0em}
\item[] None
\end{DUlineblock}

\end{description}\end{quote}

\end{description}\end{quote}

\item[{Outputs}] \leavevmode\begin{quote}\begin{description}
\item[{1.}] \leavevmode\begin{quote}\begin{description}
\item[{Name}] \leavevmode
readMe!

\item[{Description}] \leavevmode
\begin{DUlineblock}{0em}
\item[] In case of any errors, it will be shown here.
\end{DUlineblock}

\end{description}\end{quote}

\item[{2.}] \leavevmode\begin{quote}\begin{description}
\item[{Name}] \leavevmode
Units

\item[{Description}] \leavevmode
\begin{DUlineblock}{0em}
\item[] Shows the units of the curve values.
\end{DUlineblock}

\end{description}\end{quote}

\item[{3.}] \leavevmode\begin{quote}\begin{description}
\item[{Name}] \leavevmode
CurveValues

\item[{Description}] \leavevmode
\begin{DUlineblock}{0em}
\item[] Chosen curve properties values.
\end{DUlineblock}

\end{description}\end{quote}

\item[{4.}] \leavevmode\begin{quote}\begin{description}
\item[{Name}] \leavevmode
RetentionCurve

\item[{Description}] \leavevmode
\begin{DUlineblock}{0em}
\item[] Livestock Retention Curve.
\end{DUlineblock}

\end{description}\end{quote}

\end{description}\end{quote}

\end{description}\end{quote}

\sphinxstylestrong{Livestock CMF Solve}
\begin{quote}\begin{description}
\item[{Description}] \leavevmode
\begin{DUlineblock}{0em}
\item[] Solves CMF Case.
\item[] Icon art based on Vectors Market from the Noun Project.
\end{DUlineblock}

\item[{Inputs}] \leavevmode\begin{quote}\begin{description}
\item[{1.}] \leavevmode\begin{quote}\begin{description}
\item[{Name}] \leavevmode
Mesh

\item[{Description}] \leavevmode
Topography as a mesh.

\item[{Data Access}] \leavevmode
Item

\item[{Default Value}] \leavevmode
\begin{DUlineblock}{0em}
\item[] None
\end{DUlineblock}

\end{description}\end{quote}

\item[{2.}] \leavevmode\begin{quote}\begin{description}
\item[{Name}] \leavevmode
Ground

\item[{Description}] \leavevmode
Input from Livestock CMF Ground.

\item[{Data Access}] \leavevmode
List

\item[{Default Value}] \leavevmode
\begin{DUlineblock}{0em}
\item[] None
\end{DUlineblock}

\end{description}\end{quote}

\item[{3.}] \leavevmode\begin{quote}\begin{description}
\item[{Name}] \leavevmode
Weather

\item[{Description}] \leavevmode
Input from Livestock CMF Weather.

\item[{Data Access}] \leavevmode
Item

\item[{Default Value}] \leavevmode
\begin{DUlineblock}{0em}
\item[] None
\end{DUlineblock}

\end{description}\end{quote}

\item[{4.}] \leavevmode\begin{quote}\begin{description}
\item[{Name}] \leavevmode
Trees

\item[{Description}] \leavevmode
Input from Livestock CMF Tree.

\item[{Data Access}] \leavevmode
List

\item[{Default Value}] \leavevmode
\begin{DUlineblock}{0em}
\item[] None
\end{DUlineblock}

\end{description}\end{quote}

\item[{5.}] \leavevmode\begin{quote}\begin{description}
\item[{Name}] \leavevmode
Stream

\item[{Description}] \leavevmode
Input from Livestock CMF Stream. \sphinxstylestrong{Currently not working.}

\item[{Data Access}] \leavevmode
Item

\item[{Default Value}] \leavevmode
\begin{DUlineblock}{0em}
\item[] None
\end{DUlineblock}

\end{description}\end{quote}

\item[{6.}] \leavevmode\begin{quote}\begin{description}
\item[{Name}] \leavevmode
BoundaryConditions

\item[{Description}] \leavevmode
Input from Livestock CMF Boundary Condition.

\item[{Data Access}] \leavevmode
List

\item[{Default Value}] \leavevmode
\begin{DUlineblock}{0em}
\item[] None
\end{DUlineblock}

\end{description}\end{quote}

\item[{7.}] \leavevmode\begin{quote}\begin{description}
\item[{Name}] \leavevmode
SolverSettings

\item[{Description}] \leavevmode
Input from Livestock CMF Solver Settings.

\item[{Data Access}] \leavevmode
Item

\item[{Default Value}] \leavevmode
\begin{DUlineblock}{0em}
\item[] None
\end{DUlineblock}

\end{description}\end{quote}

\item[{8.}] \leavevmode\begin{quote}\begin{description}
\item[{Name}] \leavevmode
Folder

\item[{Description}] \leavevmode
Path to folder. Default is Desktop.

\item[{Data Access}] \leavevmode
Item

\item[{Default Value}] \leavevmode
\begin{DUlineblock}{0em}
\item[] os.path.join(os.environ{[}“HOMEPATH”{]}, “Desktop”)\}
\end{DUlineblock}

\end{description}\end{quote}

\item[{9.}] \leavevmode\begin{quote}\begin{description}
\item[{Name}] \leavevmode
CaseName

\item[{Description}] \leavevmode
Case name as string. Default is CMF

\item[{Data Access}] \leavevmode
Item

\item[{Default Value}] \leavevmode
\begin{DUlineblock}{0em}
\item[] CMF
\end{DUlineblock}

\end{description}\end{quote}

\item[{10.}] \leavevmode\begin{quote}\begin{description}
\item[{Name}] \leavevmode
Outputs

\item[{Description}] \leavevmode
Connect Livestock Outputs.

\item[{Data Access}] \leavevmode
Item

\item[{Default Value}] \leavevmode
\begin{DUlineblock}{0em}
\item[] None
\end{DUlineblock}

\end{description}\end{quote}

\item[{11.}] \leavevmode\begin{quote}\begin{description}
\item[{Name}] \leavevmode
Write

\item[{Description}] \leavevmode
Boolean to write files.

\item[{Data Access}] \leavevmode
Item

\item[{Default Value}] \leavevmode
\begin{DUlineblock}{0em}
\item[] False
\end{DUlineblock}

\end{description}\end{quote}

\item[{12.}] \leavevmode\begin{quote}\begin{description}
\item[{Name}] \leavevmode
Overwrite

\item[{Description}] \leavevmode
If True excising case will be overwritten. Default is set to True.

\item[{Data Access}] \leavevmode
Item

\item[{Default Value}] \leavevmode
\begin{DUlineblock}{0em}
\item[] True
\end{DUlineblock}

\end{description}\end{quote}

\item[{13.}] \leavevmode\begin{quote}\begin{description}
\item[{Name}] \leavevmode
Run

\item[{Description}] \leavevmode
\begin{DUlineblock}{0em}
\item[] Boolean to run analysis.
\item[] Analysis will be ran through SSH. Configure the connection with Livestock SSH.
\end{DUlineblock}

\item[{Data Access}] \leavevmode
Item

\item[{Default Value}] \leavevmode
\begin{DUlineblock}{0em}
\item[] False
\end{DUlineblock}

\end{description}\end{quote}

\end{description}\end{quote}

\item[{Outputs}] \leavevmode\begin{quote}\begin{description}
\item[{1.}] \leavevmode\begin{quote}\begin{description}
\item[{Name}] \leavevmode
readMe!

\item[{Description}] \leavevmode
\begin{DUlineblock}{0em}
\item[] In case of any errors, it will be shown here.
\end{DUlineblock}

\end{description}\end{quote}

\item[{2.}] \leavevmode\begin{quote}\begin{description}
\item[{Name}] \leavevmode
ResultPath

\item[{Description}] \leavevmode
\begin{DUlineblock}{0em}
\item[] Path to result files.
\end{DUlineblock}

\end{description}\end{quote}

\end{description}\end{quote}

\end{description}\end{quote}

\sphinxstylestrong{Livestock CMF Results}
\begin{quote}\begin{description}
\item[{Description}] \leavevmode
\begin{DUlineblock}{0em}
\item[] CMF Results
\end{DUlineblock}

\item[{Inputs}] \leavevmode\begin{quote}\begin{description}
\item[{1.}] \leavevmode\begin{quote}\begin{description}
\item[{Name}] \leavevmode
ResultFilePath

\item[{Description}] \leavevmode
Path to result file. Accepts output from Livestock Solve

\item[{Data Access}] \leavevmode
Item

\item[{Default Value}] \leavevmode
\begin{DUlineblock}{0em}
\item[] None
\end{DUlineblock}

\end{description}\end{quote}

\item[{2.}] \leavevmode\begin{quote}\begin{description}
\item[{Name}] \leavevmode
FetchResult

\item[{Description}] \leavevmode
\begin{DUlineblock}{0em}
\item[] Choose which result should be loaded:
\item[] 0 - Evapotranspiration
\item[] 1 - Surface water volume
\item[] 2 - Surface water flux
\item[] 3 - Heat flux
\item[] 4 - Aerodynamic resistance
\item[] 5 - Soil layer water flux
\item[] 6 - Soil layer potential
\item[] 7 - Soil layer theta
\item[] 8 - Soil layer volume
\item[] 9 - Soil layer wetness
\item[] Default is set to 0.
\end{DUlineblock}

\item[{Data Access}] \leavevmode
Item

\item[{Default Value}] \leavevmode
\begin{DUlineblock}{0em}
\item[] 0
\end{DUlineblock}

\end{description}\end{quote}

\item[{3.}] \leavevmode\begin{quote}\begin{description}
\item[{Name}] \leavevmode
SaveCSV

\item[{Description}] \leavevmode
Save the values as a csv file - Default is set to False.

\item[{Data Access}] \leavevmode
Item

\item[{Default Value}] \leavevmode
\begin{DUlineblock}{0em}
\item[] False
\end{DUlineblock}

\end{description}\end{quote}

\item[{4.}] \leavevmode\begin{quote}\begin{description}
\item[{Name}] \leavevmode
Run

\item[{Description}] \leavevmode
Run component.

\item[{Data Access}] \leavevmode
Item

\item[{Default Value}] \leavevmode
\begin{DUlineblock}{0em}
\item[] False
\end{DUlineblock}

\end{description}\end{quote}

\end{description}\end{quote}

\item[{Outputs}] \leavevmode\begin{quote}\begin{description}
\item[{1.}] \leavevmode\begin{quote}\begin{description}
\item[{Name}] \leavevmode
readMe!

\item[{Description}] \leavevmode
\begin{DUlineblock}{0em}
\item[] In case of any errors, it will be shown here.
\end{DUlineblock}

\end{description}\end{quote}

\item[{2.}] \leavevmode\begin{quote}\begin{description}
\item[{Name}] \leavevmode
Units

\item[{Description}] \leavevmode
\begin{DUlineblock}{0em}
\item[] Shows the units of the results.
\end{DUlineblock}

\end{description}\end{quote}

\item[{3.}] \leavevmode\begin{quote}\begin{description}
\item[{Name}] \leavevmode
Values

\item[{Description}] \leavevmode
\begin{DUlineblock}{0em}
\item[] List with chosen result values.
\end{DUlineblock}

\end{description}\end{quote}

\item[{4.}] \leavevmode\begin{quote}\begin{description}
\item[{Name}] \leavevmode
CSVPath

\item[{Description}] \leavevmode
\begin{DUlineblock}{0em}
\item[] Path to csv file.
\end{DUlineblock}

\end{description}\end{quote}

\end{description}\end{quote}

\end{description}\end{quote}

\sphinxstylestrong{Livestock CMF Outputs}
\begin{quote}\begin{description}
\item[{Description}] \leavevmode
CMF Outputs

\item[{Inputs}] \leavevmode\begin{quote}\begin{description}
\item[{1.}] \leavevmode\begin{quote}\begin{description}
\item[{Name}] \leavevmode
Evapotranspiration

\item[{Description}] \leavevmode
Cell evaporation - default is set to True.

\item[{Data Access}] \leavevmode
Item

\item[{Default Value}] \leavevmode
\begin{DUlineblock}{0em}
\item[] True
\end{DUlineblock}

\end{description}\end{quote}

\item[{2.}] \leavevmode\begin{quote}\begin{description}
\item[{Name}] \leavevmode
SurfaceWaterVolume

\item[{Description}] \leavevmode
Cell surface water - default is set to False.

\item[{Data Access}] \leavevmode
Item

\item[{Default Value}] \leavevmode
\begin{DUlineblock}{0em}
\item[] False
\end{DUlineblock}

\end{description}\end{quote}

\item[{3.}] \leavevmode\begin{quote}\begin{description}
\item[{Name}] \leavevmode
SurfaceWaterFlux

\item[{Description}] \leavevmode
Cell surface water flux - default is set to False.

\item[{Data Access}] \leavevmode
Item

\item[{Default Value}] \leavevmode
\begin{DUlineblock}{0em}
\item[] False
\end{DUlineblock}

\end{description}\end{quote}

\item[{4.}] \leavevmode\begin{quote}\begin{description}
\item[{Name}] \leavevmode
HeatFlux

\item[{Description}] \leavevmode
Cell surface heat flux - default is set to False.

\item[{Data Access}] \leavevmode
Item

\item[{Default Value}] \leavevmode
\begin{DUlineblock}{0em}
\item[] False
\end{DUlineblock}

\end{description}\end{quote}

\item[{5.}] \leavevmode\begin{quote}\begin{description}
\item[{Name}] \leavevmode
AerodynamicResistance

\item[{Description}] \leavevmode
Cell surface water - default is set to False.

\item[{Data Access}] \leavevmode
Item

\item[{Default Value}] \leavevmode
\begin{DUlineblock}{0em}
\item[] False
\end{DUlineblock}

\end{description}\end{quote}

\item[{6.}] \leavevmode\begin{quote}\begin{description}
\item[{Name}] \leavevmode
SurfaceWaterFlux

\item[{Description}] \leavevmode
Soil layer volumetric flux vectors - default is set to False.

\item[{Data Access}] \leavevmode
Item

\item[{Default Value}] \leavevmode
\begin{DUlineblock}{0em}
\item[] False
\end{DUlineblock}

\end{description}\end{quote}

\item[{7.}] \leavevmode\begin{quote}\begin{description}
\item[{Name}] \leavevmode
VolumetricFlux

\item[{Description}] \leavevmode
Soil layer volumetric flux vectors - default is set to False.

\item[{Data Access}] \leavevmode
Item

\item[{Default Value}] \leavevmode
\begin{DUlineblock}{0em}
\item[] False
\end{DUlineblock}

\end{description}\end{quote}

\item[{8.}] \leavevmode\begin{quote}\begin{description}
\item[{Name}] \leavevmode
Potential

\item[{Description}] \leavevmode
Soil layer total potential (Psi$_{\text{tot}}$= Psi$_{\text{M}}$+ Psi$_{\text{G}}$- default is set to False.

\item[{Data Access}] \leavevmode
Item

\item[{Default Value}] \leavevmode
\begin{DUlineblock}{0em}
\item[] False
\end{DUlineblock}

\end{description}\end{quote}

\item[{9.}] \leavevmode\begin{quote}\begin{description}
\item[{Name}] \leavevmode
Theta

\item[{Description}] \leavevmode
Soil layer volumetric water content of the layer - default is set to False.

\item[{Data Access}] \leavevmode
Item

\item[{Default Value}] \leavevmode
\begin{DUlineblock}{0em}
\item[] False
\end{DUlineblock}

\end{description}\end{quote}

\item[{10.}] \leavevmode\begin{quote}\begin{description}
\item[{Name}] \leavevmode
Volume

\item[{Description}] \leavevmode
Soil layer volume of water in the layer - default is set to True.

\item[{Data Access}] \leavevmode
Item

\item[{Default Value}] \leavevmode
\begin{DUlineblock}{0em}
\item[] True
\end{DUlineblock}

\end{description}\end{quote}

\item[{10.}] \leavevmode\begin{quote}\begin{description}
\item[{Name}] \leavevmode
Wetness

\item[{Description}] \leavevmode
Soil layer wetness of the soil (V$_{\text{volume}}$/V$_{\text{pores}}$) - default is set to False.

\item[{Data Access}] \leavevmode
Item

\item[{Default Value}] \leavevmode
\begin{DUlineblock}{0em}
\item[] False
\end{DUlineblock}

\end{description}\end{quote}

\end{description}\end{quote}

\item[{Outputs}] \leavevmode\begin{quote}\begin{description}
\item[{1.}] \leavevmode\begin{quote}\begin{description}
\item[{Name}] \leavevmode
readMe!

\item[{Description}] \leavevmode
\begin{DUlineblock}{0em}
\item[] In case of any errors, it will be shown here.
\end{DUlineblock}

\end{description}\end{quote}

\item[{2.}] \leavevmode\begin{quote}\begin{description}
\item[{Name}] \leavevmode
ChosenOutputs

\item[{Description}] \leavevmode
\begin{DUlineblock}{0em}
\item[] Shows the chosen outputs.
\end{DUlineblock}

\end{description}\end{quote}

\item[{3.}] \leavevmode\begin{quote}\begin{description}
\item[{Name}] \leavevmode
Outputs

\item[{Description}] \leavevmode
\begin{DUlineblock}{0em}
\item[] Livestock Output Data.
\end{DUlineblock}

\end{description}\end{quote}

\end{description}\end{quote}

\end{description}\end{quote}

\sphinxstylestrong{Livestock CMF Boundary Condition}
\begin{quote}\begin{description}
\item[{Description}] \leavevmode
CMF Boundary connection

\item[{Inputs}] \leavevmode\begin{quote}\begin{description}
\item[{1.}] \leavevmode\begin{quote}\begin{description}
\item[{Name}] \leavevmode
InletOrOutlet

\item[{Description}] \leavevmode
0 is inlet. 1 is outlet - default is set to 0

\item[{Data Access}] \leavevmode
Item

\item[{Default Value}] \leavevmode
\begin{DUlineblock}{0em}
\item[] 0
\end{DUlineblock}

\end{description}\end{quote}

\item[{2.}] \leavevmode\begin{quote}\begin{description}
\item[{Name}] \leavevmode
ConnectedCell

\item[{Description}] \leavevmode
Cell to connect to. Default is set to first cell.

\item[{Data Access}] \leavevmode
Item

\item[{Default Value}] \leavevmode
\begin{DUlineblock}{0em}
\item[] 0
\end{DUlineblock}

\end{description}\end{quote}

\item[{3.}] \leavevmode\begin{quote}\begin{description}
\item[{Name}] \leavevmode
ConnectedLayer

\item[{Description}] \leavevmode
Layer of cell to connect to. 0 is surface water. 1 is first layer of cell and so on.
Default is set to 0 - surface water.

\item[{Data Access}] \leavevmode
Item

\item[{Default Value}] \leavevmode
\begin{DUlineblock}{0em}
\item[] 0
\end{DUlineblock}

\end{description}\end{quote}

\item[{4.}] \leavevmode\begin{quote}\begin{description}
\item[{Name}] \leavevmode
InletFlux

\item[{Description}] \leavevmode
If inlet, then set flux in m3/day.

\item[{Data Access}] \leavevmode
List

\item[{Default Value}] \leavevmode
\begin{DUlineblock}{0em}
\item[] False
\end{DUlineblock}

\end{description}\end{quote}

\item[{5.}] \leavevmode\begin{quote}\begin{description}
\item[{Name}] \leavevmode
FlowWidth

\item[{Description}] \leavevmode
Width of the connection from cell to outlet in meters.

\item[{Data Access}] \leavevmode
Item

\item[{Default Value}] \leavevmode
\begin{DUlineblock}{0em}
\item[] None
\end{DUlineblock}

\end{description}\end{quote}

\item[{6.}] \leavevmode\begin{quote}\begin{description}
\item[{Name}] \leavevmode
OutletLocation

\item[{Description}] \leavevmode
Location of the outlet in x, y and z coordinates.

\item[{Data Access}] \leavevmode
List

\item[{Default Value}] \leavevmode
\begin{DUlineblock}{0em}
\item[] None
\end{DUlineblock}

\end{description}\end{quote}

\end{description}\end{quote}

\item[{Outputs}] \leavevmode\begin{quote}\begin{description}
\item[{1.}] \leavevmode\begin{quote}\begin{description}
\item[{Name}] \leavevmode
readMe!

\item[{Description}] \leavevmode
\begin{DUlineblock}{0em}
\item[] In case of any errors, it will be shown here.
\end{DUlineblock}

\end{description}\end{quote}

\item[{2.}] \leavevmode\begin{quote}\begin{description}
\item[{Name}] \leavevmode
BoundaryCondition

\item[{Description}] \leavevmode
\begin{DUlineblock}{0em}
\item[] Livestock Boundary Conditions.
\end{DUlineblock}

\end{description}\end{quote}

\end{description}\end{quote}

\end{description}\end{quote}

\sphinxstylestrong{Livestock CMF Solver Settings}
\begin{quote}\begin{description}
\item[{Description}] \leavevmode
Sets the solver settings for CMF Solve

\item[{Inputs}] \leavevmode\begin{quote}\begin{description}
\item[{1.}] \leavevmode\begin{quote}\begin{description}
\item[{Name}] \leavevmode
AnalysisLength

\item[{Description}] \leavevmode
Number of time steps to be taken - Default is 24

\item[{Data Access}] \leavevmode
Item

\item[{Default Value}] \leavevmode
\begin{DUlineblock}{0em}
\item[] 24
\end{DUlineblock}

\end{description}\end{quote}

\item[{2.}] \leavevmode\begin{quote}\begin{description}
\item[{Name}] \leavevmode
TimeStep

\item[{Description}] \leavevmode
Size of each time step in hours - e.g. 1/60 equals time steps of 1 min and 24 is a time step
of one day. Default is 1 hour.

\item[{Data Access}] \leavevmode
Item

\item[{Default Value}] \leavevmode
\begin{DUlineblock}{0em}
\item[] 1
\end{DUlineblock}

\end{description}\end{quote}

\item[{3.}] \leavevmode\begin{quote}\begin{description}
\item[{Name}] \leavevmode
SolverTolerance

\item[{Description}] \leavevmode
Solver tolerance - Default is 1e-8

\item[{Data Access}] \leavevmode
Item

\item[{Default Value}] \leavevmode
\begin{DUlineblock}{0em}
\item[] 10**-8
\end{DUlineblock}

\end{description}\end{quote}

\item[{4.}] \leavevmode\begin{quote}\begin{description}
\item[{Name}] \leavevmode
Verbosity

\item[{Description}] \leavevmode
\begin{DUlineblock}{0em}
\item[] Sets the verbosity of the print statement during runtime - Default is 1.
\item[] 0 - Prints only at start and end of simulation.
\item[] 1 - Prints at every time step.
\end{DUlineblock}

\item[{Data Access}] \leavevmode
Item

\item[{Default Value}] \leavevmode
\begin{DUlineblock}{0em}
\item[] 1
\end{DUlineblock}

\end{description}\end{quote}

\end{description}\end{quote}

\item[{Outputs}] \leavevmode\begin{quote}\begin{description}
\item[{1.}] \leavevmode\begin{quote}\begin{description}
\item[{Name}] \leavevmode
readMe!

\item[{Description}] \leavevmode
\begin{DUlineblock}{0em}
\item[] In case of any errors, it will be shown here.
\end{DUlineblock}

\end{description}\end{quote}

\item[{2.}] \leavevmode\begin{quote}\begin{description}
\item[{Name}] \leavevmode
SolverSettings

\item[{Description}] \leavevmode
\begin{DUlineblock}{0em}
\item[] Livestock Solver Settings.
\end{DUlineblock}

\end{description}\end{quote}

\end{description}\end{quote}

\end{description}\end{quote}

\sphinxstylestrong{Livestock CMF Surface Flux Result}
\begin{quote}\begin{description}
\item[{Description}] \leavevmode
Extract the surface flux for a mesh.

\item[{Inputs}] \leavevmode\begin{quote}\begin{description}
\item[{1.}] \leavevmode\begin{quote}\begin{description}
\item[{Name}] \leavevmode
ResultFilePath

\item[{Description}] \leavevmode
Path to result file. Accepts output from Livestock Solve

\item[{Data Access}] \leavevmode
Item

\item[{Default Value}] \leavevmode
\begin{DUlineblock}{0em}
\item[] None
\end{DUlineblock}

\end{description}\end{quote}

\item[{2.}] \leavevmode\begin{quote}\begin{description}
\item[{Name}] \leavevmode
Mesh

\item[{Description}] \leavevmode
Mesh of the case

\item[{Data Access}] \leavevmode
Item

\item[{Default Value}] \leavevmode
\begin{DUlineblock}{0em}
\item[] None
\end{DUlineblock}

\end{description}\end{quote}

\item[{3.}] \leavevmode\begin{quote}\begin{description}
\item[{Name}] \leavevmode
IncludeRunOff

\item[{Description}] \leavevmode
Include surface run-off into the surface flux vector? Default is set to True.

\item[{Data Access}] \leavevmode
Item

\item[{Default Value}] \leavevmode
\begin{DUlineblock}{0em}
\item[] True
\end{DUlineblock}

\end{description}\end{quote}

\item[{4.}] \leavevmode\begin{quote}\begin{description}
\item[{Name}] \leavevmode
IncludeRain

\item[{Description}] \leavevmode
Include rain into the surface flux vector? Default is False.

\item[{Data Access}] \leavevmode
Item

\item[{Default Value}] \leavevmode
\begin{DUlineblock}{0em}
\item[] False
\end{DUlineblock}

\end{description}\end{quote}

\item[{5.}] \leavevmode\begin{quote}\begin{description}
\item[{Name}] \leavevmode
IncludeEvapotranspiration

\item[{Description}] \leavevmode
Include evapotranspiration into the surface flux vector? Default is set to False.

\item[{Data Access}] \leavevmode
Item

\item[{Default Value}] \leavevmode
\begin{DUlineblock}{0em}
\item[] False
\end{DUlineblock}

\end{description}\end{quote}

\item[{6.}] \leavevmode\begin{quote}\begin{description}
\item[{Name}] \leavevmode
IncludeInfiltration

\item[{Description}] \leavevmode
Include infiltration into the surface flux vector? Default is False.

\item[{Data Access}] \leavevmode
Item

\item[{Default Value}] \leavevmode
\begin{DUlineblock}{0em}
\item[] False
\end{DUlineblock}

\end{description}\end{quote}

\item[{7.}] \leavevmode\begin{quote}\begin{description}
\item[{Name}] \leavevmode
SaveResult

\item[{Description}] \leavevmode
Save the values as a text file - Default is set to False.

\item[{Data Access}] \leavevmode
Item

\item[{Default Value}] \leavevmode
\begin{DUlineblock}{0em}
\item[] False
\end{DUlineblock}

\end{description}\end{quote}

\item[{8.}] \leavevmode\begin{quote}\begin{description}
\item[{Name}] \leavevmode
Run

\item[{Description}] \leavevmode
Run component. Default is False.

\item[{Data Access}] \leavevmode
Item

\item[{Default Value}] \leavevmode
\begin{DUlineblock}{0em}
\item[] False
\end{DUlineblock}

\end{description}\end{quote}

\end{description}\end{quote}

\item[{Outputs}] \leavevmode\begin{quote}\begin{description}
\item[{1.}] \leavevmode\begin{quote}\begin{description}
\item[{Name}] \leavevmode
readMe!

\item[{Description}] \leavevmode
\begin{DUlineblock}{0em}
\item[] In case of any errors, it will be shown here.
\end{DUlineblock}

\end{description}\end{quote}

\item[{2.}] \leavevmode\begin{quote}\begin{description}
\item[{Name}] \leavevmode
Unit

\item[{Description}] \leavevmode
\begin{DUlineblock}{0em}
\item[] Shows the units of the results.
\end{DUlineblock}

\end{description}\end{quote}

\item[{3.}] \leavevmode\begin{quote}\begin{description}
\item[{Name}] \leavevmode
SurfaceFluxVectors

\item[{Description}] \leavevmode
\begin{DUlineblock}{0em}
\item[] Tree with the surface flux vectors.
\end{DUlineblock}

\end{description}\end{quote}

\item[{4.}] \leavevmode\begin{quote}\begin{description}
\item[{Name}] \leavevmode
CSVPath

\item[{Description}] \leavevmode
\begin{DUlineblock}{0em}
\item[] Path to csv file.
\end{DUlineblock}

\end{description}\end{quote}

\end{description}\end{quote}

\end{description}\end{quote}


\subsection{4 \textbar{} Comfort}
\label{\detokenize{components:comfort}}
\sphinxstylestrong{Livestock New Air Conditions}
\begin{quote}\begin{description}
\item[{Description}] \leavevmode
Computes new air temperature and relative humidity

\item[{Inputs}] \leavevmode\begin{quote}\begin{description}
\item[{1.}] \leavevmode\begin{quote}\begin{description}
\item[{Name}] \leavevmode
Mesh

\item[{Description}] \leavevmode
Ground Mesh

\item[{Data Access}] \leavevmode
Item

\item[{Default Value}] \leavevmode
\begin{DUlineblock}{0em}
\item[] None
\end{DUlineblock}

\end{description}\end{quote}

\item[{2.}] \leavevmode\begin{quote}\begin{description}
\item[{Name}] \leavevmode
Evapotranspiration

\item[{Description}] \leavevmode
Evapotranspiration in m$^{\text{3}}$/day.
Each tree branch should represent one time unit, with all the cell values to that time.

\item[{Data Access}] \leavevmode
Tree

\item[{Default Value}] \leavevmode
\begin{DUlineblock}{0em}
\item[] None
\end{DUlineblock}

\end{description}\end{quote}

\item[{3.}] \leavevmode\begin{quote}\begin{description}
\item[{Name}] \leavevmode
HeatFlux

\item[{Description}] \leavevmode
HeatFlux in MJ/m$^{\text{2}}$day.
Each tree branch should represent one time unit, with all the cell values to that time.

\item[{Data Access}] \leavevmode
Tree

\item[{Default Value}] \leavevmode
\begin{DUlineblock}{0em}
\item[] None
\end{DUlineblock}

\end{description}\end{quote}

\item[{4.}] \leavevmode\begin{quote}\begin{description}
\item[{Name}] \leavevmode
AirTemperature

\item[{Description}] \leavevmode
Air temperature in C

\item[{Data Access}] \leavevmode
List

\item[{Default Value}] \leavevmode
\begin{DUlineblock}{0em}
\item[] None
\end{DUlineblock}

\end{description}\end{quote}

\item[{5.}] \leavevmode\begin{quote}\begin{description}
\item[{Name}] \leavevmode
AirRelativeHumidity

\item[{Description}] \leavevmode
Relative Humidity in -

\item[{Data Access}] \leavevmode
List

\item[{Default Value}] \leavevmode
\begin{DUlineblock}{0em}
\item[] None
\end{DUlineblock}

\end{description}\end{quote}

\item[{6.}] \leavevmode\begin{quote}\begin{description}
\item[{Name}] \leavevmode
AirBoundaryHeight

\item[{Description}] \leavevmode
Top of the air column in m. Default is set to 10m.

\item[{Data Access}] \leavevmode
Item

\item[{Default Value}] \leavevmode
\begin{DUlineblock}{0em}
\item[] 10
\end{DUlineblock}

\end{description}\end{quote}

\item[{7.}] \leavevmode\begin{quote}\begin{description}
\item[{Name}] \leavevmode
InvestigationHeight

\item[{Description}] \leavevmode
Height at which the new air temperature and relative humidity should be calculated.
Default is set to 1.1m.

\item[{Data Access}] \leavevmode
Item

\item[{Default Value}] \leavevmode
\begin{DUlineblock}{0em}
\item[] 1.1
\end{DUlineblock}

\end{description}\end{quote}

\item[{8.}] \leavevmode\begin{quote}\begin{description}
\item[{Name}] \leavevmode
CPUs

\item[{Description}] \leavevmode
Number of CPUs to perform the computations on. Default is set to 2

\item[{Data Access}] \leavevmode
Item

\item[{Default Value}] \leavevmode
\begin{DUlineblock}{0em}
\item[] 2
\end{DUlineblock}

\end{description}\end{quote}

\item[{9.}] \leavevmode\begin{quote}\begin{description}
\item[{Name}] \leavevmode
ThroughSSH

\item[{Description}] \leavevmode
If the computation should be run through SSH. Default is set to False

\item[{Data Access}] \leavevmode
Item

\item[{Default Value}] \leavevmode
\begin{DUlineblock}{0em}
\item[] False
\end{DUlineblock}

\end{description}\end{quote}

\item[{10.}] \leavevmode\begin{quote}\begin{description}
\item[{Name}] \leavevmode
Run

\item[{Description}] \leavevmode
Run the component

\item[{Data Access}] \leavevmode
Item

\item[{Default Value}] \leavevmode
\begin{DUlineblock}{0em}
\item[] False
\end{DUlineblock}

\end{description}\end{quote}

\end{description}\end{quote}

\item[{Outputs}] \leavevmode\begin{quote}\begin{description}
\item[{1.}] \leavevmode\begin{quote}\begin{description}
\item[{Name}] \leavevmode
readMe!

\item[{Description}] \leavevmode
\begin{DUlineblock}{0em}
\item[] In case of any errors, it will be shown here.
\end{DUlineblock}

\end{description}\end{quote}

\item[{2.}] \leavevmode\begin{quote}\begin{description}
\item[{Name}] \leavevmode
NewTemperature

\item[{Description}] \leavevmode
\begin{DUlineblock}{0em}
\item[] New temperature in C.
\end{DUlineblock}

\end{description}\end{quote}

\item[{3.}] \leavevmode\begin{quote}\begin{description}
\item[{Name}] \leavevmode
NewRelativeHumidity

\item[{Description}] \leavevmode
\begin{DUlineblock}{0em}
\item[] New relative humidity in -.
\end{DUlineblock}

\end{description}\end{quote}

\end{description}\end{quote}

\end{description}\end{quote}

\sphinxstylestrong{Livestock Adaptive Clothing}
\begin{quote}\begin{description}
\item[{Description}] \leavevmode
\begin{DUlineblock}{0em}
\item[] Computes the clothing isolation in clo for a given outdoor temperature.
\item[] Source: Havenith et al. - 2012 - “The UTCI-clothing model”
\end{DUlineblock}

\item[{Inputs}] \leavevmode\begin{quote}\begin{description}
\item[{1.}] \leavevmode\begin{quote}\begin{description}
\item[{Name}] \leavevmode
Temperature

\item[{Description}] \leavevmode
Temperature in C

\item[{Data Access}] \leavevmode
List

\item[{Default Value}] \leavevmode
\begin{DUlineblock}{0em}
\item[] None
\end{DUlineblock}

\end{description}\end{quote}

\end{description}\end{quote}

\item[{Outputs}] \leavevmode\begin{quote}\begin{description}
\item[{1.}] \leavevmode\begin{quote}\begin{description}
\item[{Name}] \leavevmode
readMe!

\item[{Description}] \leavevmode
\begin{DUlineblock}{0em}
\item[] In case of any errors, it will be shown here.
\end{DUlineblock}

\end{description}\end{quote}

\item[{2.}] \leavevmode\begin{quote}\begin{description}
\item[{Name}] \leavevmode
ClothingValue

\item[{Description}] \leavevmode
\begin{DUlineblock}{0em}
\item[] Calculated clothing value in clo.
\end{DUlineblock}

\end{description}\end{quote}

\end{description}\end{quote}

\end{description}\end{quote}

\sphinxstylestrong{Go Back to:}

\sphinxhref{https://ocni-dtu.github.io/}{Livestock Frontpage}

\sphinxhref{https://ocni-dtu.github.io/livestock/index.html}{Livestock PyPi}

\sphinxhref{https://ocni-dtu.github.io/livestock\_gh/index.html}{Livestock Grasshopper}


\section{Livestock Grasshopper Component Classes}
\label{\detokenize{component_classes:livestock-grasshopper-component-classes}}\label{\detokenize{component_classes:id3}}\label{\detokenize{component_classes:classes}}\label{\detokenize{component_classes::doc}}

\subsection{SuperClass}
\label{\detokenize{superclass:superclass}}\label{\detokenize{superclass::doc}}\label{\detokenize{superclass:module-livestock.components.component}}\index{livestock.components.component (module)}\index{GHComponent (class in livestock.components.component)}

\begin{fulllineitems}
\phantomsection\label{\detokenize{superclass:livestock.components.component.GHComponent}}\pysiglinewithargsret{\sphinxbfcode{class }\sphinxcode{livestock.components.component.}\sphinxbfcode{GHComponent}}{\emph{ghenv}}{}
Bases: \sphinxcode{object}
\index{add\_default\_value() (livestock.components.component.GHComponent method)}

\begin{fulllineitems}
\phantomsection\label{\detokenize{superclass:livestock.components.component.GHComponent.add_default_value}}\pysiglinewithargsret{\sphinxbfcode{add\_default\_value}}{\emph{parameter}, \emph{param\_number}}{}
Adds a default value to a parameter.
:param parameter: Parameter to add default value to
:param param\_number: Parameter number
:return: Parameter

\end{fulllineitems}

\index{add\_input\_parameter() (livestock.components.component.GHComponent method)}

\begin{fulllineitems}
\phantomsection\label{\detokenize{superclass:livestock.components.component.GHComponent.add_input_parameter}}\pysiglinewithargsret{\sphinxbfcode{add\_input\_parameter}}{\emph{input\_}}{}
Adds an input to the Grasshopper component.
:param {\color{red}\bfseries{}input\_}: Input index.

\end{fulllineitems}

\index{add\_output\_parameter() (livestock.components.component.GHComponent method)}

\begin{fulllineitems}
\phantomsection\label{\detokenize{superclass:livestock.components.component.GHComponent.add_output_parameter}}\pysiglinewithargsret{\sphinxbfcode{add\_output\_parameter}}{\emph{output\_}}{}
Adds an output to the Grasshopper component.
:param {\color{red}\bfseries{}output\_}: Output index.

\end{fulllineitems}

\index{add\_warning() (livestock.components.component.GHComponent method)}

\begin{fulllineitems}
\phantomsection\label{\detokenize{superclass:livestock.components.component.GHComponent.add_warning}}\pysiglinewithargsret{\sphinxbfcode{add\_warning}}{\emph{warning}}{}
Adds a Grasshopper warning to the component.
:param warning: Warning text.

\end{fulllineitems}

\index{config\_component() (livestock.components.component.GHComponent method)}

\begin{fulllineitems}
\phantomsection\label{\detokenize{superclass:livestock.components.component.GHComponent.config_component}}\pysiglinewithargsret{\sphinxbfcode{config\_component}}{\emph{component\_number}}{}
Sets up the component, with the following steps:
- Load component data
- Generate component data
- Generate outputs
- Generate inputs
:param component\_number: Integer with the component number

\end{fulllineitems}


\end{fulllineitems}

\index{GroundTemperature (class in livestock.components.component)}

\begin{fulllineitems}
\phantomsection\label{\detokenize{superclass:livestock.components.component.GroundTemperature}}\pysiglinewithargsret{\sphinxbfcode{class }\sphinxcode{livestock.components.component.}\sphinxbfcode{GroundTemperature}}{\emph{ghenv}}{}
Bases: {\hyperref[\detokenize{superclass:livestock.components.component.GHComponent}]{\sphinxcrossref{\sphinxcode{livestock.components.component.GHComponent}}}}

\end{fulllineitems}

\index{component\_data() (in module livestock.components.component)}

\begin{fulllineitems}
\phantomsection\label{\detokenize{superclass:livestock.components.component.component_data}}\pysiglinewithargsret{\sphinxcode{livestock.components.component.}\sphinxbfcode{component\_data}}{\emph{n}}{}
Function that reads the grasshopper component list and returns the component data

\end{fulllineitems}


\sphinxstylestrong{Go Back to:}

\sphinxhref{https://ocni-dtu.github.io/}{Livestock Frontpage}

\sphinxhref{https://ocni-dtu.github.io/livestock/index.html}{Livestock PyPi}

\sphinxhref{https://ocni-dtu.github.io/livestock\_gh/index.html}{Livestock Grasshopper}


\subsection{0 \textbar{} Miscellaneous}
\label{\detokenize{miscellaneous:id3}}\label{\detokenize{miscellaneous:miscellaneous}}\label{\detokenize{miscellaneous::doc}}\label{\detokenize{miscellaneous:module-livestock.components.misc}}\index{livestock.components.misc (module)}\index{CFDonSSH (class in livestock.components.misc)}

\begin{fulllineitems}
\phantomsection\label{\detokenize{miscellaneous:livestock.components.misc.CFDonSSH}}\pysiglinewithargsret{\sphinxbfcode{class }\sphinxcode{livestock.components.misc.}\sphinxbfcode{CFDonSSH}}{\emph{ghenv}}{}
Bases: {\hyperref[\detokenize{superclass:livestock.components.component.GHComponent}]{\sphinxcrossref{\sphinxcode{livestock.components.component.GHComponent}}}}
\index{check\_inputs() (livestock.components.misc.CFDonSSH method)}

\begin{fulllineitems}
\phantomsection\label{\detokenize{miscellaneous:livestock.components.misc.CFDonSSH.check_inputs}}\pysiglinewithargsret{\sphinxbfcode{check\_inputs}}{}{}
\end{fulllineitems}

\index{config() (livestock.components.misc.CFDonSSH method)}

\begin{fulllineitems}
\phantomsection\label{\detokenize{miscellaneous:livestock.components.misc.CFDonSSH.config}}\pysiglinewithargsret{\sphinxbfcode{config}}{}{}
\end{fulllineitems}

\index{run() (livestock.components.misc.CFDonSSH method)}

\begin{fulllineitems}
\phantomsection\label{\detokenize{miscellaneous:livestock.components.misc.CFDonSSH.run}}\pysiglinewithargsret{\sphinxbfcode{run}}{}{}
\end{fulllineitems}

\index{run\_checks() (livestock.components.misc.CFDonSSH method)}

\begin{fulllineitems}
\phantomsection\label{\detokenize{miscellaneous:livestock.components.misc.CFDonSSH.run_checks}}\pysiglinewithargsret{\sphinxbfcode{run\_checks}}{\emph{directory}, \emph{commands}, \emph{cpus}, \emph{run}}{}
\end{fulllineitems}

\index{run\_template() (livestock.components.misc.CFDonSSH method)}

\begin{fulllineitems}
\phantomsection\label{\detokenize{miscellaneous:livestock.components.misc.CFDonSSH.run_template}}\pysiglinewithargsret{\sphinxbfcode{run\_template}}{}{}
\end{fulllineitems}

\index{write() (livestock.components.misc.CFDonSSH method)}

\begin{fulllineitems}
\phantomsection\label{\detokenize{miscellaneous:livestock.components.misc.CFDonSSH.write}}\pysiglinewithargsret{\sphinxbfcode{write}}{}{}
\end{fulllineitems}


\end{fulllineitems}

\index{HourToDate (class in livestock.components.misc)}

\begin{fulllineitems}
\phantomsection\label{\detokenize{miscellaneous:livestock.components.misc.HourToDate}}\pysiglinewithargsret{\sphinxbfcode{class }\sphinxcode{livestock.components.misc.}\sphinxbfcode{HourToDate}}{\emph{ghenv}}{}
Bases: {\hyperref[\detokenize{superclass:livestock.components.component.GHComponent}]{\sphinxcrossref{\sphinxcode{livestock.components.component.GHComponent}}}}
\index{check\_inputs() (livestock.components.misc.HourToDate method)}

\begin{fulllineitems}
\phantomsection\label{\detokenize{miscellaneous:livestock.components.misc.HourToDate.check_inputs}}\pysiglinewithargsret{\sphinxbfcode{check\_inputs}}{}{}
\end{fulllineitems}

\index{config() (livestock.components.misc.HourToDate method)}

\begin{fulllineitems}
\phantomsection\label{\detokenize{miscellaneous:livestock.components.misc.HourToDate.config}}\pysiglinewithargsret{\sphinxbfcode{config}}{}{}
\end{fulllineitems}

\index{convert\_date() (livestock.components.misc.HourToDate method)}

\begin{fulllineitems}
\phantomsection\label{\detokenize{miscellaneous:livestock.components.misc.HourToDate.convert_date}}\pysiglinewithargsret{\sphinxbfcode{convert\_date}}{}{}
\end{fulllineitems}

\index{run() (livestock.components.misc.HourToDate method)}

\begin{fulllineitems}
\phantomsection\label{\detokenize{miscellaneous:livestock.components.misc.HourToDate.run}}\pysiglinewithargsret{\sphinxbfcode{run}}{}{}
\end{fulllineitems}

\index{run\_checks() (livestock.components.misc.HourToDate method)}

\begin{fulllineitems}
\phantomsection\label{\detokenize{miscellaneous:livestock.components.misc.HourToDate.run_checks}}\pysiglinewithargsret{\sphinxbfcode{run\_checks}}{\emph{hour}}{}
\end{fulllineitems}


\end{fulllineitems}

\index{PythonExecutor (class in livestock.components.misc)}

\begin{fulllineitems}
\phantomsection\label{\detokenize{miscellaneous:livestock.components.misc.PythonExecutor}}\pysiglinewithargsret{\sphinxbfcode{class }\sphinxcode{livestock.components.misc.}\sphinxbfcode{PythonExecutor}}{\emph{ghenv}}{}
Bases: {\hyperref[\detokenize{superclass:livestock.components.component.GHComponent}]{\sphinxcrossref{\sphinxcode{livestock.components.component.GHComponent}}}}
\index{check\_inputs() (livestock.components.misc.PythonExecutor method)}

\begin{fulllineitems}
\phantomsection\label{\detokenize{miscellaneous:livestock.components.misc.PythonExecutor.check_inputs}}\pysiglinewithargsret{\sphinxbfcode{check\_inputs}}{}{}
Checks inputs and raises a warning if an input is not the correct type.

\end{fulllineitems}

\index{config() (livestock.components.misc.PythonExecutor method)}

\begin{fulllineitems}
\phantomsection\label{\detokenize{miscellaneous:livestock.components.misc.PythonExecutor.config}}\pysiglinewithargsret{\sphinxbfcode{config}}{}{}
Generates the Grasshopper component.

\end{fulllineitems}

\index{run() (livestock.components.misc.PythonExecutor method)}

\begin{fulllineitems}
\phantomsection\label{\detokenize{miscellaneous:livestock.components.misc.PythonExecutor.run}}\pysiglinewithargsret{\sphinxbfcode{run}}{}{}
In case all the checks have passed the component runs.
It prints the python.exe path and creates a scriptcontext.sticky with the path.

\end{fulllineitems}

\index{run\_checks() (livestock.components.misc.PythonExecutor method)}

\begin{fulllineitems}
\phantomsection\label{\detokenize{miscellaneous:livestock.components.misc.PythonExecutor.run_checks}}\pysiglinewithargsret{\sphinxbfcode{run\_checks}}{\emph{py\_exe}}{}
Gathers the inputs and checks them.
:param py\_exe: Path to python.exe

\end{fulllineitems}


\end{fulllineitems}

\index{SSHConnection (class in livestock.components.misc)}

\begin{fulllineitems}
\phantomsection\label{\detokenize{miscellaneous:livestock.components.misc.SSHConnection}}\pysiglinewithargsret{\sphinxbfcode{class }\sphinxcode{livestock.components.misc.}\sphinxbfcode{SSHConnection}}{\emph{ghenv}}{}
Bases: {\hyperref[\detokenize{superclass:livestock.components.component.GHComponent}]{\sphinxcrossref{\sphinxcode{livestock.components.component.GHComponent}}}}
\index{check\_inputs() (livestock.components.misc.SSHConnection method)}

\begin{fulllineitems}
\phantomsection\label{\detokenize{miscellaneous:livestock.components.misc.SSHConnection.check_inputs}}\pysiglinewithargsret{\sphinxbfcode{check\_inputs}}{}{}
Checks inputs and raises a warning if an input is not the correct type.

\end{fulllineitems}

\index{config() (livestock.components.misc.SSHConnection method)}

\begin{fulllineitems}
\phantomsection\label{\detokenize{miscellaneous:livestock.components.misc.SSHConnection.config}}\pysiglinewithargsret{\sphinxbfcode{config}}{}{}
Generates the Grasshopper component.

\end{fulllineitems}

\index{run() (livestock.components.misc.SSHConnection method)}

\begin{fulllineitems}
\phantomsection\label{\detokenize{miscellaneous:livestock.components.misc.SSHConnection.run}}\pysiglinewithargsret{\sphinxbfcode{run}}{}{}
In case all the checks have passed the component runs.
It prints out the IP, Port and Username and creates a
scriptcontext.sticky all four inputs.

\end{fulllineitems}

\index{run\_checks() (livestock.components.misc.SSHConnection method)}

\begin{fulllineitems}
\phantomsection\label{\detokenize{miscellaneous:livestock.components.misc.SSHConnection.run_checks}}\pysiglinewithargsret{\sphinxbfcode{run\_checks}}{\emph{ip}, \emph{port}, \emph{user}, \emph{pw}}{}
Gathers the inputs and checks them.
:param ip: IP for SSH connection
:param port: Port for SSH connection
:param user: Username for SSH connection
:param pw: Password for SSH connection

\end{fulllineitems}


\end{fulllineitems}


\sphinxstylestrong{Go Back to:}

\sphinxhref{https://ocni-dtu.github.io/}{Livestock Frontpage}

\sphinxhref{https://ocni-dtu.github.io/livestock/index.html}{Livestock PyPi}

\sphinxhref{https://ocni-dtu.github.io/livestock\_gh/index.html}{Livestock Grasshopper}


\subsection{1 \textbar{} Geometry}
\label{\detokenize{geometry:module-livestock.components.geometry}}\label{\detokenize{geometry:id3}}\label{\detokenize{geometry::doc}}\label{\detokenize{geometry:geometry}}\index{livestock.components.geometry (module)}\index{LoadMesh (class in livestock.components.geometry)}

\begin{fulllineitems}
\phantomsection\label{\detokenize{geometry:livestock.components.geometry.LoadMesh}}\pysiglinewithargsret{\sphinxbfcode{class }\sphinxcode{livestock.components.geometry.}\sphinxbfcode{LoadMesh}}{\emph{ghenv}}{}
Bases: {\hyperref[\detokenize{superclass:livestock.components.component.GHComponent}]{\sphinxcrossref{\sphinxcode{livestock.components.component.GHComponent}}}}
\index{check\_inputs() (livestock.components.geometry.LoadMesh method)}

\begin{fulllineitems}
\phantomsection\label{\detokenize{geometry:livestock.components.geometry.LoadMesh.check_inputs}}\pysiglinewithargsret{\sphinxbfcode{check\_inputs}}{}{}
Checks inputs and raises a warning if an input is not the correct type.

\end{fulllineitems}

\index{config() (livestock.components.geometry.LoadMesh method)}

\begin{fulllineitems}
\phantomsection\label{\detokenize{geometry:livestock.components.geometry.LoadMesh.config}}\pysiglinewithargsret{\sphinxbfcode{config}}{}{}
Generates the Grasshopper component.

\end{fulllineitems}

\index{run() (livestock.components.geometry.LoadMesh method)}

\begin{fulllineitems}
\phantomsection\label{\detokenize{geometry:livestock.components.geometry.LoadMesh.run}}\pysiglinewithargsret{\sphinxbfcode{run}}{}{}
In case all the checks have passed and Load is True the component runs.
It loads the .obj file and its data if there is any.

\end{fulllineitems}

\index{run\_checks() (livestock.components.geometry.LoadMesh method)}

\begin{fulllineitems}
\phantomsection\label{\detokenize{geometry:livestock.components.geometry.LoadMesh.run_checks}}\pysiglinewithargsret{\sphinxbfcode{run\_checks}}{\emph{path}, \emph{load}}{}
Gathers the inputs and checks them.
:param path: Path where the mesh is saved.
:param load: Load the mesh or not

\end{fulllineitems}


\end{fulllineitems}

\index{SaveMesh (class in livestock.components.geometry)}

\begin{fulllineitems}
\phantomsection\label{\detokenize{geometry:livestock.components.geometry.SaveMesh}}\pysiglinewithargsret{\sphinxbfcode{class }\sphinxcode{livestock.components.geometry.}\sphinxbfcode{SaveMesh}}{\emph{ghenv}}{}
Bases: {\hyperref[\detokenize{superclass:livestock.components.component.GHComponent}]{\sphinxcrossref{\sphinxcode{livestock.components.component.GHComponent}}}}
\index{check\_inputs() (livestock.components.geometry.SaveMesh method)}

\begin{fulllineitems}
\phantomsection\label{\detokenize{geometry:livestock.components.geometry.SaveMesh.check_inputs}}\pysiglinewithargsret{\sphinxbfcode{check\_inputs}}{}{}
Checks inputs and raises a warning if an input is not the correct type.

\end{fulllineitems}

\index{config() (livestock.components.geometry.SaveMesh method)}

\begin{fulllineitems}
\phantomsection\label{\detokenize{geometry:livestock.components.geometry.SaveMesh.config}}\pysiglinewithargsret{\sphinxbfcode{config}}{}{}
Generates the Grasshopper component.

\end{fulllineitems}

\index{run() (livestock.components.geometry.SaveMesh method)}

\begin{fulllineitems}
\phantomsection\label{\detokenize{geometry:livestock.components.geometry.SaveMesh.run}}\pysiglinewithargsret{\sphinxbfcode{run}}{\emph{doc}}{}
In case all the checks have passed and save is True the component runs.
It saves the .obj file and its data if there is any.

\end{fulllineitems}

\index{run\_checks() (livestock.components.geometry.SaveMesh method)}

\begin{fulllineitems}
\phantomsection\label{\detokenize{geometry:livestock.components.geometry.SaveMesh.run_checks}}\pysiglinewithargsret{\sphinxbfcode{run\_checks}}{\emph{mesh}, \emph{data}, \emph{dir\_}, \emph{name}, \emph{save}}{}
Gathers the inputs and checks them.
:param mesh: Mesh that should be saved.
:param data: Mesh data that should be saved.
:param {\color{red}\bfseries{}dir\_}: Directory where the files should be saved.
:param name: Name for mesh.
:param save: Whether to save or not.

\end{fulllineitems}


\end{fulllineitems}


\sphinxstylestrong{Go Back to:}

\sphinxhref{https://ocni-dtu.github.io/}{Livestock Frontpage}

\sphinxhref{https://ocni-dtu.github.io/livestock/index.html}{Livestock PyPi}

\sphinxhref{https://ocni-dtu.github.io/livestock\_gh/index.html}{Livestock Grasshopper}


\subsection{3 \textbar{} CMF}
\label{\detokenize{cmf:module-livestock.components.comp_cmf}}\label{\detokenize{cmf:id3}}\label{\detokenize{cmf:cmf}}\label{\detokenize{cmf::doc}}\index{livestock.components.comp\_cmf (module)}\index{CMFGround (class in livestock.components.comp\_cmf)}

\begin{fulllineitems}
\phantomsection\label{\detokenize{cmf:livestock.components.comp_cmf.CMFGround}}\pysiglinewithargsret{\sphinxbfcode{class }\sphinxcode{livestock.components.comp\_cmf.}\sphinxbfcode{CMFGround}}{\emph{ghenv}}{}
Bases: {\hyperref[\detokenize{superclass:livestock.components.component.GHComponent}]{\sphinxcrossref{\sphinxcode{livestock.components.component.GHComponent}}}}
\index{check\_inputs() (livestock.components.comp\_cmf.CMFGround method)}

\begin{fulllineitems}
\phantomsection\label{\detokenize{cmf:livestock.components.comp_cmf.CMFGround.check_inputs}}\pysiglinewithargsret{\sphinxbfcode{check\_inputs}}{}{}
Checks inputs and raises a warning if an input is not the correct type.

\end{fulllineitems}

\index{config() (livestock.components.comp\_cmf.CMFGround method)}

\begin{fulllineitems}
\phantomsection\label{\detokenize{cmf:livestock.components.comp_cmf.CMFGround.config}}\pysiglinewithargsret{\sphinxbfcode{config}}{}{}
Generates the Grasshopper component.

\end{fulllineitems}

\index{convert\_et\_number\_to\_method() (livestock.components.comp\_cmf.CMFGround method)}

\begin{fulllineitems}
\phantomsection\label{\detokenize{cmf:livestock.components.comp_cmf.CMFGround.convert_et_number_to_method}}\pysiglinewithargsret{\sphinxbfcode{convert\_et\_number\_to\_method}}{}{}
Converts a number into a ET method.
:return: ET method name.

\end{fulllineitems}

\index{convert\_runoff\_number\_to\_method() (livestock.components.comp\_cmf.CMFGround method)}

\begin{fulllineitems}
\phantomsection\label{\detokenize{cmf:livestock.components.comp_cmf.CMFGround.convert_runoff_number_to_method}}\pysiglinewithargsret{\sphinxbfcode{convert\_runoff\_number\_to\_method}}{}{}
Converts a number into a surface run-off method.
:return: Surface run-off name

\end{fulllineitems}

\index{run() (livestock.components.comp\_cmf.CMFGround method)}

\begin{fulllineitems}
\phantomsection\label{\detokenize{cmf:livestock.components.comp_cmf.CMFGround.run}}\pysiglinewithargsret{\sphinxbfcode{run}}{}{}
In case all the checks have passed the component runs.
The component puts all the inputs into a dict and uses PassClass to pass it on.

\end{fulllineitems}

\index{run\_checks() (livestock.components.comp\_cmf.CMFGround method)}

\begin{fulllineitems}
\phantomsection\label{\detokenize{cmf:livestock.components.comp_cmf.CMFGround.run_checks}}\pysiglinewithargsret{\sphinxbfcode{run\_checks}}{\emph{layers}, \emph{retention\_curve}, \emph{vegetation\_properties}, \emph{saturated\_depth}, \emph{surface\_water}, \emph{face\_indices}, \emph{et\_method}, \emph{manning\_}, \emph{puddle}, \emph{surface\_run\_off\_method}}{}
Gathers the inputs and checks them.
:param layers: Depth of layers.
:param retention\_curve: Livestock retention curve dict.
:param vegetation\_properties: Livestock vegetation properties dict.
:param saturated\_depth: Saturated depth of the cell.
:param surface\_water: Initial surfacee water volume.
:param face\_indices: Face indices where the properties should be applied to.
:param et\_method: Evapotranspriation calculation method.
:param {\color{red}\bfseries{}manning\_}: Manning roughness.
:param puddle: Puddle depth.
:param surface\_run\_off\_method: Surface Run-off method.

\end{fulllineitems}


\end{fulllineitems}

\index{CMFInlet (class in livestock.components.comp\_cmf)}

\begin{fulllineitems}
\phantomsection\label{\detokenize{cmf:livestock.components.comp_cmf.CMFInlet}}\pysiglinewithargsret{\sphinxbfcode{class }\sphinxcode{livestock.components.comp\_cmf.}\sphinxbfcode{CMFInlet}}{\emph{ghenv}}{}
Bases: {\hyperref[\detokenize{superclass:livestock.components.component.GHComponent}]{\sphinxcrossref{\sphinxcode{livestock.components.component.GHComponent}}}}
\index{check\_inputs() (livestock.components.comp\_cmf.CMFInlet method)}

\begin{fulllineitems}
\phantomsection\label{\detokenize{cmf:livestock.components.comp_cmf.CMFInlet.check_inputs}}\pysiglinewithargsret{\sphinxbfcode{check\_inputs}}{}{}
Checks inputs and raises a warning if an input is not the correct type.

\end{fulllineitems}

\index{config() (livestock.components.comp\_cmf.CMFInlet method)}

\begin{fulllineitems}
\phantomsection\label{\detokenize{cmf:livestock.components.comp_cmf.CMFInlet.config}}\pysiglinewithargsret{\sphinxbfcode{config}}{}{}
Generates the Grasshopper component.

\end{fulllineitems}

\index{run() (livestock.components.comp\_cmf.CMFInlet method)}

\begin{fulllineitems}
\phantomsection\label{\detokenize{cmf:livestock.components.comp_cmf.CMFInlet.run}}\pysiglinewithargsret{\sphinxbfcode{run}}{}{}
In case all the checks have passed the component runs.
It runs set\_inlet().

\end{fulllineitems}

\index{run\_checks() (livestock.components.comp\_cmf.CMFInlet method)}

\begin{fulllineitems}
\phantomsection\label{\detokenize{cmf:livestock.components.comp_cmf.CMFInlet.run_checks}}\pysiglinewithargsret{\sphinxbfcode{run\_checks}}{\emph{cell}, \emph{layer}, \emph{inlet\_flux}, \emph{time\_step}}{}
Gathers the inputs and checks them.
\begin{quote}\begin{description}
\item[{Parameters}] \leavevmode\begin{itemize}
\item {} 
\sphinxstyleliteralstrong{cell} \textendash{} 

\item {} 
\sphinxstyleliteralstrong{layer} \textendash{} 

\item {} 
\sphinxstyleliteralstrong{inlet\_flux} \textendash{} 

\item {} 
\sphinxstyleliteralstrong{time\_step} \textendash{} 

\end{itemize}

\item[{Returns}] \leavevmode


\end{description}\end{quote}

\end{fulllineitems}

\index{set\_inlet() (livestock.components.comp\_cmf.CMFInlet method)}

\begin{fulllineitems}
\phantomsection\label{\detokenize{cmf:livestock.components.comp_cmf.CMFInlet.set_inlet}}\pysiglinewithargsret{\sphinxbfcode{set\_inlet}}{}{}
Constructs a dict with inlet information.

\end{fulllineitems}


\end{fulllineitems}

\index{CMFOutlet (class in livestock.components.comp\_cmf)}

\begin{fulllineitems}
\phantomsection\label{\detokenize{cmf:livestock.components.comp_cmf.CMFOutlet}}\pysiglinewithargsret{\sphinxbfcode{class }\sphinxcode{livestock.components.comp\_cmf.}\sphinxbfcode{CMFOutlet}}{\emph{ghenv}}{}
Bases: {\hyperref[\detokenize{superclass:livestock.components.component.GHComponent}]{\sphinxcrossref{\sphinxcode{livestock.components.component.GHComponent}}}}
\index{check\_inputs() (livestock.components.comp\_cmf.CMFOutlet method)}

\begin{fulllineitems}
\phantomsection\label{\detokenize{cmf:livestock.components.comp_cmf.CMFOutlet.check_inputs}}\pysiglinewithargsret{\sphinxbfcode{check\_inputs}}{}{}
Checks inputs and raises a warning if an input is not the correct type.

\end{fulllineitems}

\index{config() (livestock.components.comp\_cmf.CMFOutlet method)}

\begin{fulllineitems}
\phantomsection\label{\detokenize{cmf:livestock.components.comp_cmf.CMFOutlet.config}}\pysiglinewithargsret{\sphinxbfcode{config}}{}{}
Generates the Grasshopper component.

\end{fulllineitems}

\index{location\_to\_string() (livestock.components.comp\_cmf.CMFOutlet method)}

\begin{fulllineitems}
\phantomsection\label{\detokenize{cmf:livestock.components.comp_cmf.CMFOutlet.location_to_string}}\pysiglinewithargsret{\sphinxbfcode{location\_to\_string}}{}{}
Converts the location to a string.
\begin{quote}\begin{description}
\item[{Returns}] \leavevmode
The location as a comma separated string

\item[{Return type}] \leavevmode
str

\end{description}\end{quote}

\end{fulllineitems}

\index{run() (livestock.components.comp\_cmf.CMFOutlet method)}

\begin{fulllineitems}
\phantomsection\label{\detokenize{cmf:livestock.components.comp_cmf.CMFOutlet.run}}\pysiglinewithargsret{\sphinxbfcode{run}}{}{}
In case all the checks have passed the component runs.
It runs set\_outlet().

\end{fulllineitems}

\index{run\_checks() (livestock.components.comp\_cmf.CMFOutlet method)}

\begin{fulllineitems}
\phantomsection\label{\detokenize{cmf:livestock.components.comp_cmf.CMFOutlet.run_checks}}\pysiglinewithargsret{\sphinxbfcode{run\_checks}}{\emph{location}, \emph{cell}, \emph{layer}, \emph{type\_}, \emph{type\_parameter}}{}
Gathers the inputs and checks them.
\begin{quote}\begin{description}
\item[{Parameters}] \leavevmode\begin{itemize}
\item {} 
\sphinxstyleliteralstrong{location} \textendash{} Location of the cell

\item {} 
\sphinxstyleliteralstrong{cell} \textendash{} Cell to connect to. Default is set to first cell

\item {} 
\sphinxstyleliteralstrong{layer} \textendash{} Layer of cell to connect to. 0 is surface water.

\item {} 
\sphinxstyleliteralstrong{type} \textendash{} Type of connection from CMF Outlet Type

\item {} 
\sphinxstyleliteralstrong{type\_parameter} \textendash{} Parameter for the connection type.

\end{itemize}

\end{description}\end{quote}

\end{fulllineitems}

\index{set\_outlet() (livestock.components.comp\_cmf.CMFOutlet method)}

\begin{fulllineitems}
\phantomsection\label{\detokenize{cmf:livestock.components.comp_cmf.CMFOutlet.set_outlet}}\pysiglinewithargsret{\sphinxbfcode{set\_outlet}}{}{}
Constructs a dict with outlet information.

\end{fulllineitems}

\index{set\_outlet\_connection() (livestock.components.comp\_cmf.CMFOutlet method)}

\begin{fulllineitems}
\phantomsection\label{\detokenize{cmf:livestock.components.comp_cmf.CMFOutlet.set_outlet_connection}}\pysiglinewithargsret{\sphinxbfcode{set\_outlet\_connection}}{}{}
Constructs a dict with outlet information.

\end{fulllineitems}

\index{type\_to\_connection() (livestock.components.comp\_cmf.CMFOutlet method)}

\begin{fulllineitems}
\phantomsection\label{\detokenize{cmf:livestock.components.comp_cmf.CMFOutlet.type_to_connection}}\pysiglinewithargsret{\sphinxbfcode{type\_to\_connection}}{}{}
\end{fulllineitems}


\end{fulllineitems}

\index{CMFOutputs (class in livestock.components.comp\_cmf)}

\begin{fulllineitems}
\phantomsection\label{\detokenize{cmf:livestock.components.comp_cmf.CMFOutputs}}\pysiglinewithargsret{\sphinxbfcode{class }\sphinxcode{livestock.components.comp\_cmf.}\sphinxbfcode{CMFOutputs}}{\emph{ghenv}}{}
Bases: {\hyperref[\detokenize{superclass:livestock.components.component.GHComponent}]{\sphinxcrossref{\sphinxcode{livestock.components.component.GHComponent}}}}
\index{check\_inputs() (livestock.components.comp\_cmf.CMFOutputs method)}

\begin{fulllineitems}
\phantomsection\label{\detokenize{cmf:livestock.components.comp_cmf.CMFOutputs.check_inputs}}\pysiglinewithargsret{\sphinxbfcode{check\_inputs}}{}{}
Checks inputs and raises a warning if an input is not the correct type.

\end{fulllineitems}

\index{config() (livestock.components.comp\_cmf.CMFOutputs method)}

\begin{fulllineitems}
\phantomsection\label{\detokenize{cmf:livestock.components.comp_cmf.CMFOutputs.config}}\pysiglinewithargsret{\sphinxbfcode{config}}{}{}
Generates the Grasshopper component.

\end{fulllineitems}

\index{run() (livestock.components.comp\_cmf.CMFOutputs method)}

\begin{fulllineitems}
\phantomsection\label{\detokenize{cmf:livestock.components.comp_cmf.CMFOutputs.run}}\pysiglinewithargsret{\sphinxbfcode{run}}{}{}
In case all the checks have passed the component runs.
set\_outputs() are run and passed on with PassClass.

\end{fulllineitems}

\index{run\_checks() (livestock.components.comp\_cmf.CMFOutputs method)}

\begin{fulllineitems}
\phantomsection\label{\detokenize{cmf:livestock.components.comp_cmf.CMFOutputs.run_checks}}\pysiglinewithargsret{\sphinxbfcode{run\_checks}}{\emph{evapo\_trans}, \emph{surface\_water\_volume}, \emph{surface\_water\_flux}, \emph{heat\_flux}, \emph{aero\_res}, \emph{three\_d\_flux}, \emph{potential}, \emph{theta}, \emph{volume}, \emph{wetness}}{}
Gathers the inputs and checks them.
:param evapo\_trans: Whether to include evapotranspiration or not.
:param surface\_water\_volume: Whether to include surface water volume or not.
:param surface\_water\_flux: Whether to include surface water flux or not.
:param heat\_flux: Whether to include surface heat flux or not.
:param aero\_res: Whether to include aerodynamic resistance or not.
:param three\_d\_flux: Whether to include soil water flux or not.
:param potential: Whether to include soil potential or not.
:param theta: Whether to include soil theta or not.
:param volume: Whether to include soil water volume or not.
:param wetness: Whether to include soil wetness or not.

\end{fulllineitems}

\index{set\_outputs() (livestock.components.comp\_cmf.CMFOutputs method)}

\begin{fulllineitems}
\phantomsection\label{\detokenize{cmf:livestock.components.comp_cmf.CMFOutputs.set_outputs}}\pysiglinewithargsret{\sphinxbfcode{set\_outputs}}{}{}
Convertes the wanted outputs into a dict
:return: output dict.

\end{fulllineitems}


\end{fulllineitems}

\index{CMFResults (class in livestock.components.comp\_cmf)}

\begin{fulllineitems}
\phantomsection\label{\detokenize{cmf:livestock.components.comp_cmf.CMFResults}}\pysiglinewithargsret{\sphinxbfcode{class }\sphinxcode{livestock.components.comp\_cmf.}\sphinxbfcode{CMFResults}}{\emph{ghenv}}{}
Bases: {\hyperref[\detokenize{superclass:livestock.components.component.GHComponent}]{\sphinxcrossref{\sphinxcode{livestock.components.component.GHComponent}}}}
\index{check\_inputs() (livestock.components.comp\_cmf.CMFResults method)}

\begin{fulllineitems}
\phantomsection\label{\detokenize{cmf:livestock.components.comp_cmf.CMFResults.check_inputs}}\pysiglinewithargsret{\sphinxbfcode{check\_inputs}}{}{}
Checks inputs and raises a warning if an input is not the correct type.

\end{fulllineitems}

\index{config() (livestock.components.comp\_cmf.CMFResults method)}

\begin{fulllineitems}
\phantomsection\label{\detokenize{cmf:livestock.components.comp_cmf.CMFResults.config}}\pysiglinewithargsret{\sphinxbfcode{config}}{}{}
Generates the Grasshopper component.

\end{fulllineitems}

\index{delete\_files() (livestock.components.comp\_cmf.CMFResults method)}

\begin{fulllineitems}
\phantomsection\label{\detokenize{cmf:livestock.components.comp_cmf.CMFResults.delete_files}}\pysiglinewithargsret{\sphinxbfcode{delete\_files}}{\emph{csv\_path}}{}
Delete the helper files.

\end{fulllineitems}

\index{load\_result\_csv() (livestock.components.comp\_cmf.CMFResults method)}

\begin{fulllineitems}
\phantomsection\label{\detokenize{cmf:livestock.components.comp_cmf.CMFResults.load_result_csv}}\pysiglinewithargsret{\sphinxbfcode{load\_result\_csv}}{\emph{path}}{}
Loads the csv file containing the wanted results.
:param path: Csv file path
:return: The results

\end{fulllineitems}

\index{process\_xml() (livestock.components.comp\_cmf.CMFResults method)}

\begin{fulllineitems}
\phantomsection\label{\detokenize{cmf:livestock.components.comp_cmf.CMFResults.process_xml}}\pysiglinewithargsret{\sphinxbfcode{process\_xml}}{}{}
Processes the xml result file and extracts the wanted information and saves it as a csv file.
:return: csv file path.

\end{fulllineitems}

\index{run() (livestock.components.comp\_cmf.CMFResults method)}

\begin{fulllineitems}
\phantomsection\label{\detokenize{cmf:livestock.components.comp_cmf.CMFResults.run}}\pysiglinewithargsret{\sphinxbfcode{run}}{}{}
In case all the checks have passed and run is True the component runs.
Following functions are run:
set\_units()
process\_xml()
load\_result\_csv()
delete\_files()
The results are converted into a Grasshopper Tree structure.

\end{fulllineitems}

\index{run\_checks() (livestock.components.comp\_cmf.CMFResults method)}

\begin{fulllineitems}
\phantomsection\label{\detokenize{cmf:livestock.components.comp_cmf.CMFResults.run_checks}}\pysiglinewithargsret{\sphinxbfcode{run\_checks}}{\emph{path}, \emph{fetch\_result}, \emph{save}, \emph{run}}{}
Gathers the inputs and checks them.
:param path: Result path
:param fetch\_result: Which result to fetch
:param save: Whether to save the csv file or not.
:param run: Whether to run the component or not.

\end{fulllineitems}

\index{set\_units() (livestock.components.comp\_cmf.CMFResults method)}

\begin{fulllineitems}
\phantomsection\label{\detokenize{cmf:livestock.components.comp_cmf.CMFResults.set_units}}\pysiglinewithargsret{\sphinxbfcode{set\_units}}{}{}
Function to organize the output units.

\end{fulllineitems}


\end{fulllineitems}

\index{CMFRetentionCurve (class in livestock.components.comp\_cmf)}

\begin{fulllineitems}
\phantomsection\label{\detokenize{cmf:livestock.components.comp_cmf.CMFRetentionCurve}}\pysiglinewithargsret{\sphinxbfcode{class }\sphinxcode{livestock.components.comp\_cmf.}\sphinxbfcode{CMFRetentionCurve}}{\emph{ghenv}}{}
Bases: {\hyperref[\detokenize{superclass:livestock.components.component.GHComponent}]{\sphinxcrossref{\sphinxcode{livestock.components.component.GHComponent}}}}
\index{check\_inputs() (livestock.components.comp\_cmf.CMFRetentionCurve method)}

\begin{fulllineitems}
\phantomsection\label{\detokenize{cmf:livestock.components.comp_cmf.CMFRetentionCurve.check_inputs}}\pysiglinewithargsret{\sphinxbfcode{check\_inputs}}{}{}
Checks inputs and raises a warning if an input is not the correct type.

\end{fulllineitems}

\index{config() (livestock.components.comp\_cmf.CMFRetentionCurve method)}

\begin{fulllineitems}
\phantomsection\label{\detokenize{cmf:livestock.components.comp_cmf.CMFRetentionCurve.config}}\pysiglinewithargsret{\sphinxbfcode{config}}{}{}
Generates the Grasshopper component.

\end{fulllineitems}

\index{load\_csv() (livestock.components.comp\_cmf.CMFRetentionCurve method)}

\begin{fulllineitems}
\phantomsection\label{\detokenize{cmf:livestock.components.comp_cmf.CMFRetentionCurve.load_csv}}\pysiglinewithargsret{\sphinxbfcode{load\_csv}}{}{}
Loads a csv file with the retention curve data.

\end{fulllineitems}

\index{load\_retention\_curve() (livestock.components.comp\_cmf.CMFRetentionCurve method)}

\begin{fulllineitems}
\phantomsection\label{\detokenize{cmf:livestock.components.comp_cmf.CMFRetentionCurve.load_retention_curve}}\pysiglinewithargsret{\sphinxbfcode{load\_retention\_curve}}{}{}
Loads the retention curve data and converts it into a order dict.
If any property is to be overwritten it is also done in this function.

\end{fulllineitems}

\index{run() (livestock.components.comp\_cmf.CMFRetentionCurve method)}

\begin{fulllineitems}
\phantomsection\label{\detokenize{cmf:livestock.components.comp_cmf.CMFRetentionCurve.run}}\pysiglinewithargsret{\sphinxbfcode{run}}{}{}
In case all the checks have passed the component runs.
Loads the retention curve data and passes it on with PassClass.

\end{fulllineitems}

\index{run\_checks() (livestock.components.comp\_cmf.CMFRetentionCurve method)}

\begin{fulllineitems}
\phantomsection\label{\detokenize{cmf:livestock.components.comp_cmf.CMFRetentionCurve.run_checks}}\pysiglinewithargsret{\sphinxbfcode{run\_checks}}{\emph{soil\_index}, \emph{k\_sat}, \emph{phi}, \emph{alpha}, \emph{n}, \emph{m}, \emph{l}}{}
Gathers the inputs and checks them.
:param soil\_index: Soil index.
:param k\_sat: Ksat
:param phi: Phi
:param alpha: Alpha
:param n: N
:param m: M
:param l: L

\end{fulllineitems}


\end{fulllineitems}

\index{CMFSolve (class in livestock.components.comp\_cmf)}

\begin{fulllineitems}
\phantomsection\label{\detokenize{cmf:livestock.components.comp_cmf.CMFSolve}}\pysiglinewithargsret{\sphinxbfcode{class }\sphinxcode{livestock.components.comp\_cmf.}\sphinxbfcode{CMFSolve}}{\emph{ghenv}}{}
Bases: {\hyperref[\detokenize{superclass:livestock.components.component.GHComponent}]{\sphinxcrossref{\sphinxcode{livestock.components.component.GHComponent}}}}
\index{check\_inputs() (livestock.components.comp\_cmf.CMFSolve method)}

\begin{fulllineitems}
\phantomsection\label{\detokenize{cmf:livestock.components.comp_cmf.CMFSolve.check_inputs}}\pysiglinewithargsret{\sphinxbfcode{check\_inputs}}{}{}
Checks inputs and raises a warning if an input is not the correct type.

\end{fulllineitems}

\index{check\_results() (livestock.components.comp\_cmf.CMFSolve method)}

\begin{fulllineitems}
\phantomsection\label{\detokenize{cmf:livestock.components.comp_cmf.CMFSolve.check_results}}\pysiglinewithargsret{\sphinxbfcode{check\_results}}{}{}
Checks if the result files exists and then copies them form the ssh folder to the case folder.
If not then a warning is raised.

\end{fulllineitems}

\index{config() (livestock.components.comp\_cmf.CMFSolve method)}

\begin{fulllineitems}
\phantomsection\label{\detokenize{cmf:livestock.components.comp_cmf.CMFSolve.config}}\pysiglinewithargsret{\sphinxbfcode{config}}{}{}
Generates the Grasshopper component.

\end{fulllineitems}

\index{do\_case() (livestock.components.comp\_cmf.CMFSolve method)}

\begin{fulllineitems}
\phantomsection\label{\detokenize{cmf:livestock.components.comp_cmf.CMFSolve.do_case}}\pysiglinewithargsret{\sphinxbfcode{do\_case}}{}{}
Spawns a new subprocess, that runs the ssh template.

\end{fulllineitems}

\index{run() (livestock.components.comp\_cmf.CMFSolve method)}

\begin{fulllineitems}
\phantomsection\label{\detokenize{cmf:livestock.components.comp_cmf.CMFSolve.run}}\pysiglinewithargsret{\sphinxbfcode{run}}{\emph{doc}}{}
In case all the checks have passed and write\_case is True the component writes the case files.
If all checks have passed and run\_case is True the simulation is started.

\end{fulllineitems}

\index{run\_checks() (livestock.components.comp\_cmf.CMFSolve method)}

\begin{fulllineitems}
\phantomsection\label{\detokenize{cmf:livestock.components.comp_cmf.CMFSolve.run_checks}}\pysiglinewithargsret{\sphinxbfcode{run\_checks}}{\emph{mesh}, \emph{ground}, \emph{weather}, \emph{trees}, \emph{stream}, \emph{boundary\_conditions}, \emph{solver\_settings}, \emph{folder}, \emph{name}, \emph{outputs}, \emph{write}, \emph{overwrite}, \emph{run}}{}
Gathers the inputs and checks them.
:param mesh: Project Mesh
:param ground: Livestock Ground dict
:param weather: Livestock Weather dict
:param trees: Livestock Tree dict
:param stream: Livestock Stream dict
:param boundary\_conditions: Livestock Boundary Condition dict
:param solver\_settings: Livestock Solver settings dict
:param folder: Case folder
:param name: Case name
:param outputs: Livestock Outputs dict
:param write: Whether to write or not
:param overwrite: Overwrite exsiting files
:param run: Whether to run or not.

\end{fulllineitems}

\index{update\_case\_path() (livestock.components.comp\_cmf.CMFSolve method)}

\begin{fulllineitems}
\phantomsection\label{\detokenize{cmf:livestock.components.comp_cmf.CMFSolve.update_case_path}}\pysiglinewithargsret{\sphinxbfcode{update\_case\_path}}{}{}
Updates the case folder path.

\end{fulllineitems}

\index{write() (livestock.components.comp\_cmf.CMFSolve method)}

\begin{fulllineitems}
\phantomsection\label{\detokenize{cmf:livestock.components.comp_cmf.CMFSolve.write}}\pysiglinewithargsret{\sphinxbfcode{write}}{\emph{doc}}{}
Writes the needed files.
:param doc: Grasshopper document.

\end{fulllineitems}


\end{fulllineitems}

\index{CMFSolverSettings (class in livestock.components.comp\_cmf)}

\begin{fulllineitems}
\phantomsection\label{\detokenize{cmf:livestock.components.comp_cmf.CMFSolverSettings}}\pysiglinewithargsret{\sphinxbfcode{class }\sphinxcode{livestock.components.comp\_cmf.}\sphinxbfcode{CMFSolverSettings}}{\emph{ghenv}}{}
Bases: {\hyperref[\detokenize{superclass:livestock.components.component.GHComponent}]{\sphinxcrossref{\sphinxcode{livestock.components.component.GHComponent}}}}
\index{check\_inputs() (livestock.components.comp\_cmf.CMFSolverSettings method)}

\begin{fulllineitems}
\phantomsection\label{\detokenize{cmf:livestock.components.comp_cmf.CMFSolverSettings.check_inputs}}\pysiglinewithargsret{\sphinxbfcode{check\_inputs}}{}{}
Checks inputs and raises a warning if an input is not the correct type.

\end{fulllineitems}

\index{config() (livestock.components.comp\_cmf.CMFSolverSettings method)}

\begin{fulllineitems}
\phantomsection\label{\detokenize{cmf:livestock.components.comp_cmf.CMFSolverSettings.config}}\pysiglinewithargsret{\sphinxbfcode{config}}{}{}
Generates the Grasshopper component.

\end{fulllineitems}

\index{run() (livestock.components.comp\_cmf.CMFSolverSettings method)}

\begin{fulllineitems}
\phantomsection\label{\detokenize{cmf:livestock.components.comp_cmf.CMFSolverSettings.run}}\pysiglinewithargsret{\sphinxbfcode{run}}{}{}
In case all the checks have passed the component runs.
Constructs a solver settings dict, prints it and passes it on with PassClass.

\end{fulllineitems}

\index{run\_checks() (livestock.components.comp\_cmf.CMFSolverSettings method)}

\begin{fulllineitems}
\phantomsection\label{\detokenize{cmf:livestock.components.comp_cmf.CMFSolverSettings.run_checks}}\pysiglinewithargsret{\sphinxbfcode{run\_checks}}{\emph{length}, \emph{time\_step}, \emph{tolerance}, \emph{verbosity}}{}
Gathers the inputs and checks them.
:param length: Number of time steps to be taken.
:param time\_step: Size of time step.
:param tolerance: Solver tolerance.
:param verbosity: Solver verbosity.

\end{fulllineitems}


\end{fulllineitems}

\index{CMFSurfaceFluxResult (class in livestock.components.comp\_cmf)}

\begin{fulllineitems}
\phantomsection\label{\detokenize{cmf:livestock.components.comp_cmf.CMFSurfaceFluxResult}}\pysiglinewithargsret{\sphinxbfcode{class }\sphinxcode{livestock.components.comp\_cmf.}\sphinxbfcode{CMFSurfaceFluxResult}}{\emph{ghenv}}{}
Bases: {\hyperref[\detokenize{superclass:livestock.components.component.GHComponent}]{\sphinxcrossref{\sphinxcode{livestock.components.component.GHComponent}}}}
\index{check\_inputs() (livestock.components.comp\_cmf.CMFSurfaceFluxResult method)}

\begin{fulllineitems}
\phantomsection\label{\detokenize{cmf:livestock.components.comp_cmf.CMFSurfaceFluxResult.check_inputs}}\pysiglinewithargsret{\sphinxbfcode{check\_inputs}}{}{}
Checks inputs and raises a warning if an input is not the correct type.

\end{fulllineitems}

\index{config() (livestock.components.comp\_cmf.CMFSurfaceFluxResult method)}

\begin{fulllineitems}
\phantomsection\label{\detokenize{cmf:livestock.components.comp_cmf.CMFSurfaceFluxResult.config}}\pysiglinewithargsret{\sphinxbfcode{config}}{}{}
Generates the Grasshopper component.

\end{fulllineitems}

\index{delete\_files() (livestock.components.comp\_cmf.CMFSurfaceFluxResult method)}

\begin{fulllineitems}
\phantomsection\label{\detokenize{cmf:livestock.components.comp_cmf.CMFSurfaceFluxResult.delete_files}}\pysiglinewithargsret{\sphinxbfcode{delete\_files}}{}{}
Deletes the helper files.

\end{fulllineitems}

\index{load\_result() (livestock.components.comp\_cmf.CMFSurfaceFluxResult method)}

\begin{fulllineitems}
\phantomsection\label{\detokenize{cmf:livestock.components.comp_cmf.CMFSurfaceFluxResult.load_result}}\pysiglinewithargsret{\sphinxbfcode{load\_result}}{}{}
Loads the results.

\end{fulllineitems}

\index{run() (livestock.components.comp\_cmf.CMFSurfaceFluxResult method)}

\begin{fulllineitems}
\phantomsection\label{\detokenize{cmf:livestock.components.comp_cmf.CMFSurfaceFluxResult.run}}\pysiglinewithargsret{\sphinxbfcode{run}}{}{}
In case all the checks have passed and Write is True the component writes the component files.
In case all the checks have passed the component runs.
The following functions are run:
write()
run\_template()
load\_results()
delete\_files()
The results are converted into a Grasshopper Tree structure.

\end{fulllineitems}

\index{run\_checks() (livestock.components.comp\_cmf.CMFSurfaceFluxResult method)}

\begin{fulllineitems}
\phantomsection\label{\detokenize{cmf:livestock.components.comp_cmf.CMFSurfaceFluxResult.run_checks}}\pysiglinewithargsret{\sphinxbfcode{run\_checks}}{\emph{path}, \emph{mesh}, \emph{run\_off}, \emph{rain}, \emph{evapo}, \emph{infiltration}, \emph{save}, \emph{write}, \emph{run}}{}
Gathers the inputs and checks them.
:param path: Path for result file.
:param mesh: Case mesh
:param run\_off: Whether to include run-off or not.
:param rain: Whether to include rain or not.
:param evapo: Whether to include evapotranspiration or not.
:param infiltration: Whether to include infiltration or not.
:param save: Save result to file or not
:param write: Write component files or not
:param run: Run component or not.

\end{fulllineitems}

\index{run\_template() (livestock.components.comp\_cmf.CMFSurfaceFluxResult method)}

\begin{fulllineitems}
\phantomsection\label{\detokenize{cmf:livestock.components.comp_cmf.CMFSurfaceFluxResult.run_template}}\pysiglinewithargsret{\sphinxbfcode{run\_template}}{}{}
Spawns a subprocess that runs the template.

\end{fulllineitems}

\index{write() (livestock.components.comp\_cmf.CMFSurfaceFluxResult method)}

\begin{fulllineitems}
\phantomsection\label{\detokenize{cmf:livestock.components.comp_cmf.CMFSurfaceFluxResult.write}}\pysiglinewithargsret{\sphinxbfcode{write}}{}{}
Writes the wanted information from the mesh and the flux configurations.

\end{fulllineitems}


\end{fulllineitems}

\index{CMFSyntheticTree (class in livestock.components.comp\_cmf)}

\begin{fulllineitems}
\phantomsection\label{\detokenize{cmf:livestock.components.comp_cmf.CMFSyntheticTree}}\pysiglinewithargsret{\sphinxbfcode{class }\sphinxcode{livestock.components.comp\_cmf.}\sphinxbfcode{CMFSyntheticTree}}{\emph{ghenv}}{}
Bases: {\hyperref[\detokenize{superclass:livestock.components.component.GHComponent}]{\sphinxcrossref{\sphinxcode{livestock.components.component.GHComponent}}}}
\index{check\_inputs() (livestock.components.comp\_cmf.CMFSyntheticTree method)}

\begin{fulllineitems}
\phantomsection\label{\detokenize{cmf:livestock.components.comp_cmf.CMFSyntheticTree.check_inputs}}\pysiglinewithargsret{\sphinxbfcode{check\_inputs}}{}{}
Checks inputs and raises a warning if an input is not the correct type.

\end{fulllineitems}

\index{compute\_tree() (livestock.components.comp\_cmf.CMFSyntheticTree method)}

\begin{fulllineitems}
\phantomsection\label{\detokenize{cmf:livestock.components.comp_cmf.CMFSyntheticTree.compute_tree}}\pysiglinewithargsret{\sphinxbfcode{compute\_tree}}{}{}
Selects the correct tree property. It computes the property information and stores it as a ordered dict.

\end{fulllineitems}

\index{config() (livestock.components.comp\_cmf.CMFSyntheticTree method)}

\begin{fulllineitems}
\phantomsection\label{\detokenize{cmf:livestock.components.comp_cmf.CMFSyntheticTree.config}}\pysiglinewithargsret{\sphinxbfcode{config}}{}{}
Generates the Grasshopper component.

\end{fulllineitems}

\index{load\_csv() (livestock.components.comp\_cmf.CMFSyntheticTree method)}

\begin{fulllineitems}
\phantomsection\label{\detokenize{cmf:livestock.components.comp_cmf.CMFSyntheticTree.load_csv}}\pysiglinewithargsret{\sphinxbfcode{load\_csv}}{}{}
Loads a csv file with the tree properties.

\end{fulllineitems}

\index{run() (livestock.components.comp\_cmf.CMFSyntheticTree method)}

\begin{fulllineitems}
\phantomsection\label{\detokenize{cmf:livestock.components.comp_cmf.CMFSyntheticTree.run}}\pysiglinewithargsret{\sphinxbfcode{run}}{}{}
In case all the checks have passed the component runs.
It runs the function compute\_tree. Creates a dict and passes it on with PassClass.

\end{fulllineitems}

\index{run\_checks() (livestock.components.comp\_cmf.CMFSyntheticTree method)}

\begin{fulllineitems}
\phantomsection\label{\detokenize{cmf:livestock.components.comp_cmf.CMFSyntheticTree.run_checks}}\pysiglinewithargsret{\sphinxbfcode{run\_checks}}{\emph{face\_index}, \emph{tree\_type}, \emph{height}}{}
Gathers the inputs and checks them.
:param face\_index: Mesh face index.
:param tree\_type: Tree type.
:param height: Tree height.

\end{fulllineitems}


\end{fulllineitems}

\index{CMFVegetationProperties (class in livestock.components.comp\_cmf)}

\begin{fulllineitems}
\phantomsection\label{\detokenize{cmf:livestock.components.comp_cmf.CMFVegetationProperties}}\pysiglinewithargsret{\sphinxbfcode{class }\sphinxcode{livestock.components.comp\_cmf.}\sphinxbfcode{CMFVegetationProperties}}{\emph{ghenv}}{}
Bases: {\hyperref[\detokenize{superclass:livestock.components.component.GHComponent}]{\sphinxcrossref{\sphinxcode{livestock.components.component.GHComponent}}}}
\index{check\_inputs() (livestock.components.comp\_cmf.CMFVegetationProperties method)}

\begin{fulllineitems}
\phantomsection\label{\detokenize{cmf:livestock.components.comp_cmf.CMFVegetationProperties.check_inputs}}\pysiglinewithargsret{\sphinxbfcode{check\_inputs}}{}{}
Checks inputs and raises a warning if an input is not the correct type.

\end{fulllineitems}

\index{config() (livestock.components.comp\_cmf.CMFVegetationProperties method)}

\begin{fulllineitems}
\phantomsection\label{\detokenize{cmf:livestock.components.comp_cmf.CMFVegetationProperties.config}}\pysiglinewithargsret{\sphinxbfcode{config}}{}{}
Generates the Grasshopper component.

\end{fulllineitems}

\index{load\_csv() (livestock.components.comp\_cmf.CMFVegetationProperties method)}

\begin{fulllineitems}
\phantomsection\label{\detokenize{cmf:livestock.components.comp_cmf.CMFVegetationProperties.load_csv}}\pysiglinewithargsret{\sphinxbfcode{load\_csv}}{}{}
Loads a csv file with the vegetation properties.

\end{fulllineitems}

\index{pick\_property() (livestock.components.comp\_cmf.CMFVegetationProperties method)}

\begin{fulllineitems}
\phantomsection\label{\detokenize{cmf:livestock.components.comp_cmf.CMFVegetationProperties.pick_property}}\pysiglinewithargsret{\sphinxbfcode{pick\_property}}{}{}
Selects the correct vegetation property. And stores it as a ordered dict.

\end{fulllineitems}

\index{run() (livestock.components.comp\_cmf.CMFVegetationProperties method)}

\begin{fulllineitems}
\phantomsection\label{\detokenize{cmf:livestock.components.comp_cmf.CMFVegetationProperties.run}}\pysiglinewithargsret{\sphinxbfcode{run}}{}{}
In case all the checks have passed the component runs.
It run pick\_properties()
And passes on the property dict with PassClass.

\end{fulllineitems}

\index{run\_checks() (livestock.components.comp\_cmf.CMFVegetationProperties method)}

\begin{fulllineitems}
\phantomsection\label{\detokenize{cmf:livestock.components.comp_cmf.CMFVegetationProperties.run_checks}}\pysiglinewithargsret{\sphinxbfcode{run\_checks}}{\emph{property\_}}{}
Gathers the inputs and checks them.
:param {\color{red}\bfseries{}property\_}: Property index.

\end{fulllineitems}


\end{fulllineitems}

\index{CMFWeather (class in livestock.components.comp\_cmf)}

\begin{fulllineitems}
\phantomsection\label{\detokenize{cmf:livestock.components.comp_cmf.CMFWeather}}\pysiglinewithargsret{\sphinxbfcode{class }\sphinxcode{livestock.components.comp\_cmf.}\sphinxbfcode{CMFWeather}}{\emph{ghenv}}{}
Bases: {\hyperref[\detokenize{superclass:livestock.components.component.GHComponent}]{\sphinxcrossref{\sphinxcode{livestock.components.component.GHComponent}}}}
\index{check\_inputs() (livestock.components.comp\_cmf.CMFWeather method)}

\begin{fulllineitems}
\phantomsection\label{\detokenize{cmf:livestock.components.comp_cmf.CMFWeather.check_inputs}}\pysiglinewithargsret{\sphinxbfcode{check\_inputs}}{}{}
Checks inputs and raises a warning if an input is not the correct type.

\end{fulllineitems}

\index{config() (livestock.components.comp\_cmf.CMFWeather method)}

\begin{fulllineitems}
\phantomsection\label{\detokenize{cmf:livestock.components.comp_cmf.CMFWeather.config}}\pysiglinewithargsret{\sphinxbfcode{config}}{}{}
Generates the Grasshopper component.

\end{fulllineitems}

\index{convert\_cloud\_cover() (livestock.components.comp\_cmf.CMFWeather method)}

\begin{fulllineitems}
\phantomsection\label{\detokenize{cmf:livestock.components.comp_cmf.CMFWeather.convert_cloud_cover}}\pysiglinewithargsret{\sphinxbfcode{convert\_cloud\_cover}}{}{}
Converts cloud cover to sun shine fraction.
sun shine = 1 - cloud cover
:return: list with sun shine fractions.

\end{fulllineitems}

\index{convert\_location() (livestock.components.comp\_cmf.CMFWeather method)}

\begin{fulllineitems}
\phantomsection\label{\detokenize{cmf:livestock.components.comp_cmf.CMFWeather.convert_location}}\pysiglinewithargsret{\sphinxbfcode{convert\_location}}{}{}
Extracts information from a Ladybug Tools location
:return: Latitude, longitude and time zone

\end{fulllineitems}

\index{convert\_radiation\_unit() (livestock.components.comp\_cmf.CMFWeather method)}

\begin{fulllineitems}
\phantomsection\label{\detokenize{cmf:livestock.components.comp_cmf.CMFWeather.convert_radiation_unit}}\pysiglinewithargsret{\sphinxbfcode{convert\_radiation\_unit}}{}{}
Converts radiation from W/m2 to MJ/(m2*day)
1 W/m2 =\textgreater{} 60s*60min*24hours/10\textasciicircum{}6 =\textgreater{} MJ/(m2*day)
1 W/m2 = 0.0864 MJ/(m2*day)

\end{fulllineitems}

\index{convert\_rain\_unit() (livestock.components.comp\_cmf.CMFWeather method)}

\begin{fulllineitems}
\phantomsection\label{\detokenize{cmf:livestock.components.comp_cmf.CMFWeather.convert_rain_unit}}\pysiglinewithargsret{\sphinxbfcode{convert\_rain\_unit}}{}{}
Converts rain from mm/h to mm/day
1 mm/h = 24 mm/day

\end{fulllineitems}

\index{match\_cell\_count() (livestock.components.comp\_cmf.CMFWeather method)}

\begin{fulllineitems}
\phantomsection\label{\detokenize{cmf:livestock.components.comp_cmf.CMFWeather.match_cell_count}}\pysiglinewithargsret{\sphinxbfcode{match\_cell\_count}}{\emph{weather\_parameter}}{}
Checks whether a whether a weather parameter has the correct number of sublists,
so they matches the number of cells. Then converts it into a dict with a list for each cell.
:param weather\_parameter: Weather parameter to check.
:return: Corrected weather parameter as dict.

\end{fulllineitems}

\index{print\_weather\_lengths() (livestock.components.comp\_cmf.CMFWeather method)}

\begin{fulllineitems}
\phantomsection\label{\detokenize{cmf:livestock.components.comp_cmf.CMFWeather.print_weather_lengths}}\pysiglinewithargsret{\sphinxbfcode{print\_weather\_lengths}}{}{}
Prints the length of each weather parameter.

\end{fulllineitems}

\index{run() (livestock.components.comp\_cmf.CMFWeather method)}

\begin{fulllineitems}
\phantomsection\label{\detokenize{cmf:livestock.components.comp_cmf.CMFWeather.run}}\pysiglinewithargsret{\sphinxbfcode{run}}{}{}
In case all the checks have passed the component runs.
The following functions are run:
print\_weather\_lengths()
convert\_cloud\_cover()
convert\_radiation\_unit()
convert\_location()
A weather dict is created an passes on with PassClass.

\end{fulllineitems}

\index{run\_checks() (livestock.components.comp\_cmf.CMFWeather method)}

\begin{fulllineitems}
\phantomsection\label{\detokenize{cmf:livestock.components.comp_cmf.CMFWeather.run_checks}}\pysiglinewithargsret{\sphinxbfcode{run\_checks}}{\emph{temp}, \emph{wind}, \emph{rel\_hum}, \emph{cloud\_cover}, \emph{global\_radiation}, \emph{rain}, \emph{ground\_temp}, \emph{location}, \emph{face\_count}}{}
Gathers the inputs and checks them.
:param temp: Temperature
:param wind: Wind speed
:param rel\_hum: Relative humidity
:param cloud\_cover: Cloud cover
:param global\_radiation: Global radiation
:param rain: Rain
:param ground\_temp: Ground temperature.
:param location: Ladybug Tool location
:param face\_count: Number of mesh faces in project

\end{fulllineitems}


\end{fulllineitems}


\sphinxstylestrong{Go Back to:}

\sphinxhref{https://ocni-dtu.github.io/}{Livestock Frontpage}

\sphinxhref{https://ocni-dtu.github.io/livestock/index.html}{Livestock PyPi}

\sphinxhref{https://ocni-dtu.github.io/livestock\_gh/index.html}{Livestock Grasshopper}


\subsection{4 \textbar{} Comfort}
\label{\detokenize{comfort:module-livestock.components.comfort}}\label{\detokenize{comfort:id3}}\label{\detokenize{comfort::doc}}\label{\detokenize{comfort:comfort}}\index{livestock.components.comfort (module)}\index{AdaptiveClothing (class in livestock.components.comfort)}

\begin{fulllineitems}
\phantomsection\label{\detokenize{comfort:livestock.components.comfort.AdaptiveClothing}}\pysiglinewithargsret{\sphinxbfcode{class }\sphinxcode{livestock.components.comfort.}\sphinxbfcode{AdaptiveClothing}}{\emph{ghenv}}{}
Bases: {\hyperref[\detokenize{superclass:livestock.components.component.GHComponent}]{\sphinxcrossref{\sphinxcode{livestock.components.component.GHComponent}}}}
\index{check\_inputs() (livestock.components.comfort.AdaptiveClothing method)}

\begin{fulllineitems}
\phantomsection\label{\detokenize{comfort:livestock.components.comfort.AdaptiveClothing.check_inputs}}\pysiglinewithargsret{\sphinxbfcode{check\_inputs}}{}{}
Checks inputs and raises a warning if an input is not the correct type.

\end{fulllineitems}

\index{config() (livestock.components.comfort.AdaptiveClothing method)}

\begin{fulllineitems}
\phantomsection\label{\detokenize{comfort:livestock.components.comfort.AdaptiveClothing.config}}\pysiglinewithargsret{\sphinxbfcode{config}}{}{}
Generates the Grasshopper component.

\end{fulllineitems}

\index{insulation\_clothing() (livestock.components.comfort.AdaptiveClothing method)}

\begin{fulllineitems}
\phantomsection\label{\detokenize{comfort:livestock.components.comfort.AdaptiveClothing.insulation_clothing}}\pysiglinewithargsret{\sphinxbfcode{insulation\_clothing}}{}{}
Calculates the clothing isolation in clo for a given outdoor temperature.
Source: Havenith et al. - 2012 - “The UTCI-clothing model”

\end{fulllineitems}

\index{run() (livestock.components.comfort.AdaptiveClothing method)}

\begin{fulllineitems}
\phantomsection\label{\detokenize{comfort:livestock.components.comfort.AdaptiveClothing.run}}\pysiglinewithargsret{\sphinxbfcode{run}}{}{}
In case all the checks have passed and run is True the component runs.
It runs the insulation\_clothing() function.

\end{fulllineitems}

\index{run\_checks() (livestock.components.comfort.AdaptiveClothing method)}

\begin{fulllineitems}
\phantomsection\label{\detokenize{comfort:livestock.components.comfort.AdaptiveClothing.run_checks}}\pysiglinewithargsret{\sphinxbfcode{run\_checks}}{\emph{temp}}{}
Gathers the inputs and checks them.
:param temp: Outdoor temperature.

\end{fulllineitems}


\end{fulllineitems}


\sphinxstylestrong{Go Back to:}

\sphinxhref{https://ocni-dtu.github.io/}{Livestock Frontpage}

\sphinxhref{https://ocni-dtu.github.io/livestock/index.html}{Livestock PyPi}

\sphinxhref{https://ocni-dtu.github.io/livestock\_gh/index.html}{Livestock Grasshopper}

\sphinxstylestrong{Go Back to:}

\sphinxhref{https://ocni-dtu.github.io/}{Livestock Frontpage}

\sphinxhref{https://ocni-dtu.github.io/livestock/index.html}{Livestock PyPi}

\sphinxhref{https://ocni-dtu.github.io/livestock\_gh/index.html}{Livestock Grasshopper}


\section{Livestock Grasshopper Lib}
\label{\detokenize{lib:lib}}\label{\detokenize{lib:id3}}\label{\detokenize{lib:livestock-grasshopper-lib}}\label{\detokenize{lib::doc}}

\subsection{Geometry}
\label{\detokenize{lib:module-livestock.lib.geometry}}\label{\detokenize{lib:geometry}}\index{livestock.lib.geometry (module)}\index{bake() (in module livestock.lib.geometry)}

\begin{fulllineitems}
\phantomsection\label{\detokenize{lib:livestock.lib.geometry.bake}}\pysiglinewithargsret{\sphinxcode{livestock.lib.geometry.}\sphinxbfcode{bake}}{\emph{geo}, \emph{doc}}{}
Bakes geometry from Grasshopper
\begin{quote}\begin{description}
\item[{Parameters}] \leavevmode\begin{itemize}
\item {} 
\sphinxstyleliteralstrong{geo} \textendash{} Geometry to bake

\item {} 
\sphinxstyleliteralstrong{doc} \textendash{} Grasshopper doc

\end{itemize}

\item[{Returns}] \leavevmode
Rhino ID

\end{description}\end{quote}

\end{fulllineitems}

\index{export() (in module livestock.lib.geometry)}

\begin{fulllineitems}
\phantomsection\label{\detokenize{lib:livestock.lib.geometry.export}}\pysiglinewithargsret{\sphinxcode{livestock.lib.geometry.}\sphinxbfcode{export}}{\emph{ids}, \emph{file\_path}, \emph{file\_name}, \emph{file\_type}, \emph{doc}}{}
Exports Rhino geometry to a file.
\begin{quote}\begin{description}
\item[{Parameters}] \leavevmode\begin{itemize}
\item {} 
\sphinxstyleliteralstrong{ids} \textendash{} Geometry ID

\item {} 
\sphinxstyleliteralstrong{file\_path} \textendash{} File directory

\item {} 
\sphinxstyleliteralstrong{file\_name} \textendash{} File name

\item {} 
\sphinxstyleliteralstrong{file\_type} \textendash{} File extension

\item {} 
\sphinxstyleliteralstrong{doc} \textendash{} Grasshopper document

\end{itemize}

\item[{Returns}] \leavevmode
True on succes.

\end{description}\end{quote}

\end{fulllineitems}

\index{bake\_export\_delete() (in module livestock.lib.geometry)}

\begin{fulllineitems}
\phantomsection\label{\detokenize{lib:livestock.lib.geometry.bake_export_delete}}\pysiglinewithargsret{\sphinxcode{livestock.lib.geometry.}\sphinxbfcode{bake\_export\_delete}}{\emph{geo}, \emph{file\_path}, \emph{file\_name}, \emph{file\_type}, \emph{doc}}{}
Bakes and exports Grasshopper geometry.
\begin{quote}\begin{description}
\item[{Parameters}] \leavevmode\begin{itemize}
\item {} 
\sphinxstyleliteralstrong{geo} \textendash{} Grasshopper geometry.

\item {} 
\sphinxstyleliteralstrong{file\_path} \textendash{} File directory

\item {} 
\sphinxstyleliteralstrong{file\_name} \textendash{} File name

\item {} 
\sphinxstyleliteralstrong{file\_type} \textendash{} File extension.

\item {} 
\sphinxstyleliteralstrong{doc} \textendash{} Grasshopper doument

\end{itemize}

\end{description}\end{quote}

\end{fulllineitems}

\index{import\_obj() (in module livestock.lib.geometry)}

\begin{fulllineitems}
\phantomsection\label{\detokenize{lib:livestock.lib.geometry.import_obj}}\pysiglinewithargsret{\sphinxcode{livestock.lib.geometry.}\sphinxbfcode{import\_obj}}{\emph{path}}{}
Reads a .obj file and converts it into a Rhino Mesh.
\begin{quote}\begin{description}
\item[{Parameters}] \leavevmode
\sphinxstyleliteralstrong{path} \textendash{} path including file name and file extension (.obj)

\item[{Returns}] \leavevmode
Rhino Mesh

\end{description}\end{quote}

\end{fulllineitems}

\index{load\_points() (in module livestock.lib.geometry)}

\begin{fulllineitems}
\phantomsection\label{\detokenize{lib:livestock.lib.geometry.load_points}}\pysiglinewithargsret{\sphinxcode{livestock.lib.geometry.}\sphinxbfcode{load\_points}}{\emph{path\_and\_file}}{}
Loads a text file containing points
\begin{quote}\begin{description}
\item[{Parameters}] \leavevmode
\sphinxstyleliteralstrong{path\_and\_file} (\sphinxstyleliteralemphasis{object}) \textendash{} 

\end{description}\end{quote}

\end{fulllineitems}

\index{make\_curves\_from\_points() (in module livestock.lib.geometry)}

\begin{fulllineitems}
\phantomsection\label{\detokenize{lib:livestock.lib.geometry.make_curves_from_points}}\pysiglinewithargsret{\sphinxcode{livestock.lib.geometry.}\sphinxbfcode{make\_curves\_from\_points}}{\emph{points}}{}
Converts a list of points to a 5-degree polynomium curve.

\end{fulllineitems}

\index{load\_mesh\_data() (in module livestock.lib.geometry)}

\begin{fulllineitems}
\phantomsection\label{\detokenize{lib:livestock.lib.geometry.load_mesh_data}}\pysiglinewithargsret{\sphinxcode{livestock.lib.geometry.}\sphinxbfcode{load\_mesh\_data}}{\emph{path}}{}
Load additional data for a mesh.
:param path: Path for mesh file.
:return: Data

\end{fulllineitems}

\index{get\_mesh\_faces() (in module livestock.lib.geometry)}

\begin{fulllineitems}
\phantomsection\label{\detokenize{lib:livestock.lib.geometry.get_mesh_faces}}\pysiglinewithargsret{\sphinxcode{livestock.lib.geometry.}\sphinxbfcode{get\_mesh\_faces}}{\emph{mesh}}{}
Takes a mesh and convert its faces into individual meshes.
\begin{quote}\begin{description}
\item[{Parameters}] \leavevmode
\sphinxstyleliteralstrong{mesh} \textendash{} mesh

\item[{Returns}] \leavevmode
list of “face” meshes

\end{description}\end{quote}

\end{fulllineitems}



\subsection{Miscellaneous}
\label{\detokenize{lib:miscellaneous}}\label{\detokenize{lib:module-livestock.lib.misc}}\index{livestock.lib.misc (module)}\index{tree\_to\_list() (in module livestock.lib.misc)}

\begin{fulllineitems}
\phantomsection\label{\detokenize{lib:livestock.lib.misc.tree_to_list}}\pysiglinewithargsret{\sphinxcode{livestock.lib.misc.}\sphinxbfcode{tree\_to\_list}}{\emph{input\_}, \emph{retrieve\_base=\textless{}function \textless{}lambda\textgreater{}\textgreater{}}}{}
Returns a list representation of a Grasshopper DataTree

\end{fulllineitems}

\index{list\_to\_tree() (in module livestock.lib.misc)}

\begin{fulllineitems}
\phantomsection\label{\detokenize{lib:livestock.lib.misc.list_to_tree}}\pysiglinewithargsret{\sphinxcode{livestock.lib.misc.}\sphinxbfcode{list\_to\_tree}}{\emph{input\_, none\_and\_holes=True, source={[}0{]}}}{}
Transforms nestings of lists or tuples to a Grasshopper DataTree

\end{fulllineitems}

\index{PassClass() (in module livestock.lib.misc)}

\begin{fulllineitems}
\phantomsection\label{\detokenize{lib:livestock.lib.misc.PassClass}}\pysiglinewithargsret{\sphinxcode{livestock.lib.misc.}\sphinxbfcode{PassClass}}{\emph{pyClass}, \emph{name}}{}
Pass a class from one Grasshopper component to another.

\end{fulllineitems}

\index{write\_file() (in module livestock.lib.misc)}

\begin{fulllineitems}
\phantomsection\label{\detokenize{lib:livestock.lib.misc.write_file}}\pysiglinewithargsret{\sphinxcode{livestock.lib.misc.}\sphinxbfcode{write\_file}}{\emph{text}, \emph{path}, \emph{name}, \emph{file\_type='txt'}}{}
Writes a text file.
:param text: Text to write.
:param path: Directory to save it to.
:param name: File name.
:param file\_type: File extension.

\end{fulllineitems}

\index{decompose\_ladybug\_location() (in module livestock.lib.misc)}

\begin{fulllineitems}
\phantomsection\label{\detokenize{lib:livestock.lib.misc.decompose_ladybug_location}}\pysiglinewithargsret{\sphinxcode{livestock.lib.misc.}\sphinxbfcode{decompose\_ladybug\_location}}{\emph{\_location}}{}
Decompose a Ladybug Tools location in to a tuple.
\begin{quote}\begin{description}
\item[{Parameters}] \leavevmode
\sphinxstyleliteralstrong{\_location} (\sphinxstyleliteralemphasis{str}) \textendash{} Ladybug Location.

\item[{Returns}] \leavevmode
Tuple with location values.

\item[{Return type}] \leavevmode
tuple

\end{description}\end{quote}

\end{fulllineitems}

\index{get\_python\_exe() (in module livestock.lib.misc)}

\begin{fulllineitems}
\phantomsection\label{\detokenize{lib:livestock.lib.misc.get_python_exe}}\pysiglinewithargsret{\sphinxcode{livestock.lib.misc.}\sphinxbfcode{get\_python\_exe}}{}{}
Collects the python.exe path from a sticky.
\begin{quote}\begin{description}
\item[{Returns}] \leavevmode
The python path.

\item[{Return type}] \leavevmode
str

\end{description}\end{quote}

\end{fulllineitems}



\subsection{SSH}
\label{\detokenize{lib:ssh}}\label{\detokenize{lib:module-livestock.lib.ssh}}\index{livestock.lib.ssh (module)}\index{get\_ssh() (in module livestock.lib.ssh)}

\begin{fulllineitems}
\phantomsection\label{\detokenize{lib:livestock.lib.ssh.get_ssh}}\pysiglinewithargsret{\sphinxcode{livestock.lib.ssh.}\sphinxbfcode{get\_ssh}}{}{}
Extracts the SSH information from a stricky

\end{fulllineitems}

\index{clean\_ssh\_folder() (in module livestock.lib.ssh)}

\begin{fulllineitems}
\phantomsection\label{\detokenize{lib:livestock.lib.ssh.clean_ssh_folder}}\pysiglinewithargsret{\sphinxcode{livestock.lib.ssh.}\sphinxbfcode{clean\_ssh\_folder}}{}{}
Cleans the livestock/ssh folder on the C drive.

\end{fulllineitems}

\index{write\_ssh\_commands() (in module livestock.lib.ssh)}

\begin{fulllineitems}
\phantomsection\label{\detokenize{lib:livestock.lib.ssh.write_ssh_commands}}\pysiglinewithargsret{\sphinxcode{livestock.lib.ssh.}\sphinxbfcode{write\_ssh\_commands}}{\emph{ssh\_dict}}{}
Write the files need for Livestock SSH connection to work.
:param ssh\_dict: Dictionary with all SSH information. Needs to be on the following form:
\{‘ip’: string, ‘user’: string, ‘port’: string, ‘password’: ‘string’, ‘file\_transfer’: list of strings,
‘file\_run’: list of strings, ‘file\_return’: list of strings, ‘template’: string\}
:return:

\end{fulllineitems}

\index{clean\_local\_folder() (in module livestock.lib.ssh)}

\begin{fulllineitems}
\phantomsection\label{\detokenize{lib:livestock.lib.ssh.clean_local_folder}}\pysiglinewithargsret{\sphinxcode{livestock.lib.ssh.}\sphinxbfcode{clean\_local\_folder}}{}{}
Cleans the livestock/local folder on the C drive.

\end{fulllineitems}



\subsection{Templates}
\label{\detokenize{lib:module-livestock.lib.templates}}\label{\detokenize{lib:templates}}\index{livestock.lib.templates (module)}\index{pick\_template() (in module livestock.lib.templates)}

\begin{fulllineitems}
\phantomsection\label{\detokenize{lib:livestock.lib.templates.pick_template}}\pysiglinewithargsret{\sphinxcode{livestock.lib.templates.}\sphinxbfcode{pick\_template}}{\emph{template\_name}, \emph{path}}{}
Writes a template given a template name and path to write it to.
:param template\_name: Template name.
:param path: Path to save it to.

\end{fulllineitems}

\index{drain\_mesh\_template() (in module livestock.lib.templates)}

\begin{fulllineitems}
\phantomsection\label{\detokenize{lib:livestock.lib.templates.drain_mesh_template}}\pysiglinewithargsret{\sphinxcode{livestock.lib.templates.}\sphinxbfcode{drain\_mesh\_template}}{\emph{path}}{}
Writes the template for the drain mesh function.
:param path: Path to write it to.

\end{fulllineitems}

\index{ssh\_template() (in module livestock.lib.templates)}

\begin{fulllineitems}
\phantomsection\label{\detokenize{lib:livestock.lib.templates.ssh_template}}\pysiglinewithargsret{\sphinxcode{livestock.lib.templates.}\sphinxbfcode{ssh\_template}}{\emph{path}}{}
Writes the ssh template.
:param path: Path to write it to.

\end{fulllineitems}

\index{cmf\_template() (in module livestock.lib.templates)}

\begin{fulllineitems}
\phantomsection\label{\detokenize{lib:livestock.lib.templates.cmf_template}}\pysiglinewithargsret{\sphinxcode{livestock.lib.templates.}\sphinxbfcode{cmf\_template}}{\emph{path}}{}
Writes the CMF template.
:param path: Path to write it to.

\end{fulllineitems}

\index{process\_cmf\_results() (in module livestock.lib.templates)}

\begin{fulllineitems}
\phantomsection\label{\detokenize{lib:livestock.lib.templates.process_cmf_results}}\pysiglinewithargsret{\sphinxcode{livestock.lib.templates.}\sphinxbfcode{process\_cmf\_results}}{\emph{path}}{}
Writes the CMF result lookup template.
:param path: Path to write it to.

\end{fulllineitems}

\index{process\_cmf\_surface\_results() (in module livestock.lib.templates)}

\begin{fulllineitems}
\phantomsection\label{\detokenize{lib:livestock.lib.templates.process_cmf_surface_results}}\pysiglinewithargsret{\sphinxcode{livestock.lib.templates.}\sphinxbfcode{process\_cmf\_surface\_results}}{\emph{path}}{}
Writes the CMF surface result template.
:param path: Path to write it to.

\end{fulllineitems}

\index{new\_air\_conditions() (in module livestock.lib.templates)}

\begin{fulllineitems}
\phantomsection\label{\detokenize{lib:livestock.lib.templates.new_air_conditions}}\pysiglinewithargsret{\sphinxcode{livestock.lib.templates.}\sphinxbfcode{new\_air\_conditions}}{\emph{path}}{}
Writes the new air condition template.
\begin{quote}\begin{description}
\item[{Parameters}] \leavevmode
\sphinxstyleliteralstrong{path} (\sphinxstyleliteralemphasis{str}) \textendash{} Path to write it to.

\item[{Returns}] \leavevmode
File name of the template

\item[{Return type}] \leavevmode
str

\end{description}\end{quote}

\end{fulllineitems}


\sphinxstylestrong{Go Back to:}

\sphinxhref{https://ocni-dtu.github.io/}{Livestock Frontpage}

\sphinxhref{https://ocni-dtu.github.io/livestock/index.html}{Livestock PyPi}

\sphinxhref{https://ocni-dtu.github.io/livestock\_gh/index.html}{Livestock Grasshopper}


\chapter{Documentation for the PyPI Package:}
\label{\detokenize{index:documentation-for-the-pypi-package}}\label{\detokenize{index:id3}}\begin{itemize}
\item {} 
\sphinxhref{https://ocni-dtu.github.io/livestock/air.html}{Livestock Air}

\item {} 
\sphinxhref{https://ocni-dtu.github.io/livestock/geometry.html}{Livestock Geometry}

\item {} 
\sphinxhref{https://ocni-dtu.github.io/livestock/hydrology.html}{Livestock Hydrology}

\item {} 
\sphinxhref{https://ocni-dtu.github.io/livestock/ssh.html}{Livestock SSH}

\end{itemize}

\sphinxstylestrong{Go Back to:}

\sphinxhref{https://ocni-dtu.github.io/}{Main Page}


\chapter{Indices and tables}
\label{\detokenize{index:indices-and-tables}}\label{\detokenize{index:id5}}\begin{itemize}
\item {} 
\DUrole{xref,std,std-ref}{genindex}

\item {} 
\DUrole{xref,std,std-ref}{modindex}

\item {} 
\DUrole{xref,std,std-ref}{search}

\end{itemize}


\renewcommand{\indexname}{Python Module Index}
\begin{sphinxtheindex}
\def\bigletter#1{{\Large\sffamily#1}\nopagebreak\vspace{1mm}}
\bigletter{l}
\item {\sphinxstyleindexentry{livestock.components.comfort}}\sphinxstyleindexpageref{comfort:\detokenize{module-livestock.components.comfort}}
\item {\sphinxstyleindexentry{livestock.components.comp\_cmf}}\sphinxstyleindexpageref{cmf:\detokenize{module-livestock.components.comp_cmf}}
\item {\sphinxstyleindexentry{livestock.components.component}}\sphinxstyleindexpageref{superclass:\detokenize{module-livestock.components.component}}
\item {\sphinxstyleindexentry{livestock.components.geometry}}\sphinxstyleindexpageref{geometry:\detokenize{module-livestock.components.geometry}}
\item {\sphinxstyleindexentry{livestock.components.misc}}\sphinxstyleindexpageref{miscellaneous:\detokenize{module-livestock.components.misc}}
\item {\sphinxstyleindexentry{livestock.lib.geometry}}\sphinxstyleindexpageref{lib:\detokenize{module-livestock.lib.geometry}}
\item {\sphinxstyleindexentry{livestock.lib.misc}}\sphinxstyleindexpageref{lib:\detokenize{module-livestock.lib.misc}}
\item {\sphinxstyleindexentry{livestock.lib.ssh}}\sphinxstyleindexpageref{lib:\detokenize{module-livestock.lib.ssh}}
\item {\sphinxstyleindexentry{livestock.lib.templates}}\sphinxstyleindexpageref{lib:\detokenize{module-livestock.lib.templates}}
\end{sphinxtheindex}

\renewcommand{\indexname}{Index}
\printindex
\end{document}